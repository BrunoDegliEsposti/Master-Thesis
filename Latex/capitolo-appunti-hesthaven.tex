\chapter{Metodi numerici per le equazioni di Eulero}

Questa è una prima stesura del capitolo \dots

\section{Appunti sul capitolo 4 di Hesthaven}

Consideriamo le griglie spaziali e temporali
\[
x_j = hj = \frac{L}{N}j
\qquad t^n = kn = \frac{T}{K}k
\]
Consideriamo una sola legge di conservazione in una sola dimensione spaziale.
Uno schema è in forma di conservazione se
\[
U_j^{n+1} = U_j^n - \frac{k}{h} (F_{j+1/2}^n - F_{j-1/2}^n)
\]
con flussi numerici
\[
F_{j+1/2}^n = F(U_{j-p}^n, \dots, U_{j+q}^n)
\qquad F_{j-1/2}^n = F(U_{j-p-1}^n, \dots, U_{j+q-1}^n)
\]
Ipotesi utili: flusso consistente
\[
F(u,\dots,u) = f(u)
\]
e flusso (localmente?) lischitziano in ogni sua variabile.

\begin{teor} Uno schema in forma conservativa con flusso consistente
è conservativo a livello discreto.
\end{teor}

\begin{coro} Uno schema in forma conservativa con flusso consistente
propaga le discontinuità alla velocità giusta perché rispetta la
condizione di Rankine-Hugoniot.
\end{coro}

\begin{teor}[Lax-Wendroff] Se schema in forma conservativa con flusso consistente
e lischitziano converge a una soluzione con variazione totale limitata in $L^1$,
allora tale soluzione è una soluzione debole della legge di conservazione.
\end{teor}

\noindent Ma è anche soluzione di entropia? Non è detto.
Consideriamo l'equazione di Burgers
\[
u_t + (u^2)_x = 0
\qquad x \in [-1,1]
\]
con condizioni iniziali
\[
u_0(x) = \begin{cases}
-1 & \text{se $x < 0$} \\
 1 & \text{se $x \geq 0$}
\end{cases}
\]
È noto che la soluzione giusta, cioè quella debole di entropia,
è l'onda di rarefazione
\[
u(t) = \begin{cases}
-1 & \text{se $x \leq -2t$} \\
x/(2t) & \text{se $-2t < x < 2t$} \\
 1 & \text{se $x \geq 2t$}
\end{cases}
\]
Tuttavia, se scegliamo come flusso numerico il flusso di Roe
\[
F_R(u,v) = \begin{cases}
f(u) & \text{se $s \geq 0$} \\
f(v) & \text{se $s < 0$}
\end{cases}
\qquad s = \frac{f(u)-f(v)}{u-v}
\]
otteniamo come soluzione $u(t) \equiv u_0$, la quale è certamente
una soluzione debole, ma non quella desiderata.
Introduciamo quindi il concetto di schema numerico \emph{monotono}.
Sia $G$ la funzione tale che
\[
U_j^{n+1} = G(U_{j-p-1}^n, \dots, U_{j+q}^n)
\]
Per esempio, se $p = 0$, $q = 1$ e lo schema è in forma di conservazione,
allora
\begin{gather*}
U_j^{n+1}
= U_j^n - \frac{k}{h} (F(U_j^n, U_{j+1}^n) - F(U_{j-1}^n, U_j^n)) \\
G(U_{j-1}^n,U_j^n,U_{j+1}^n)
= U_j^n - \frac{k}{h} (F(U_j^n, U_{j+1}^n) - F(U_{j-1}^n, U_j^n))
\end{gather*}
Uno schema numerico si dice \emph{monotono} se la funzione $G$
è crescente in ogni suo argomento.

\begin{teor}
Il flusso numerico di Lax–Friedrichs
\[
F_{LF}(u,v) = \frac{f(u)+f(v)}{2} - \frac{\alpha}{2} (v-u)
\]
dà luogo a uno schema monotono, a patto che
\[
\alpha \geq \max \Bigl\{ \abs{f'(u)},\abs{f'(v)} \Bigr\}
\qquad 1 - \frac{k}{h} \left( \frac{f'(u)-f'(v)}{2} + \alpha \right) \geq 0
\]
Osserviamo che la seconda diseguaglianza è sicuramente soddisfatta
se il passo temporale è sufficientemente piccolo rispetto a quello spaziale.
\end{teor}

\begin{proof}
Basta scrivere $G$ e imporre che tutte le derivate parziali siano positive.
\end{proof}

\begin{teor}
Sia $G$ uno schema numerico monotono. Allora
\begin{enumerate}
\item $U_j^n \leq V_j^n$ per ogni $j$ implica $G(U^n)_j \leq G(V^n)_j$ per ogni $j$.
\item Sia $S_j = \{ j-p-1, \dots, j+q \}$. Allora la soluzione numerica
	soddisfa un principio di massimo locale su ogni stencil $S_j$:
	\[
	\min_{i \in S_j} U_i^n \leq G(U^n)_j \leq \max_{i \in S_j} U_i^n
	\]
\item Lo schema è contrattivo in norma $\ell^1$:
	\[
	\norm{G(U^n)-G(V^n)}_1 \leq \norm{U^n-V^n}_1
	\]
\item Lo schema non aumenta la variazione totale a ogni iterazione:
	\[
	\TV(G(U^n)) \leq \TV(U^n)
	\]
\end{enumerate}
\end{teor}

\noindent Dunque la soluzione numerica ottenuta con uno schema
monotono preserva tutti gli aspetti qualitativi della soluzione analitica.
In particolare, lo schema è stabile e non produce oscillazioni spurie.

\begin{teor}
Sia $G$ uno schema numerico monotono e supponiamo che le soluzioni numeriche
$U^n$ convergano, nel limite di $k \to 0, h \to 0$, a una soluzione debole
della legge di conservazione. Allora tale soluzione soddisfa tutte le
condizioni di entropia.
\end{teor}

\noindent Uno schema numerico $G$ si dice che \emph{preserva la monotonicità}
a ogni iterazione se $U_j^n \leq U_{j+1}^n$ per ogni $j$ implica che
$U_j^{n+1} \leq U_{j+1}^{n+1}$ per ogni $j$.

\begin{teor}
Uno schema numerico che non aumenta la variazione totale a ogni iterazione
preserva la monotonicità.
\end{teor}

\noindent Abbiamo quindi dimostrato la seguente catena di implicazioni
per le proprietà di uno schema numerico:
\begin{equation} \label{eq:catena}
\text{Monotonia}
\Rightarrow \text{Non aumenta $\norm{\cdot}_1$}
\Rightarrow \text{Non aumenta TV}
\Rightarrow \text{Preserva la monotonicità}
\end{equation}
La catena di implicazioni addirittura si chiude in alcuni casi particolari.
Consideriamo ad esempio l'equazione delle onde lineare del primo ordine
\begin{equation} \label{eq:onde-lineare-1D}
\partial_t u + \partial_x u = 0
\end{equation}
e uno schema numerico \emph{lineare} per la sua soluzione
\[
U_j^{n+1} = \sum_{i = -p-1}^q c_i(k,h) U_{j+i}^n
\]
I coefficienti $c_i$ dipendono quindi solo dalla discretizzazione scelta,
ma non da $x$, $t$ o $U$. Osserviamo che gli schemi visti finora si scrivono
tutti in questa forma.

\begin{teor}
Uno schema numerico lineare per la soluzione dell'equazione \eqref{eq:onde-lineare-1D}
è monotono se e solo se preserva la monotonicità.
\end{teor}

Purtroppo, l'ipotesi di monotonicità non è priva di controindicazioni:
essendo molto stringente, non permette di ottenere metodi con ordine
di convergenza superiore al primo:

\begin{teor}
Uno schema numerico monotono ha ordine di convergenza al più uno.
\end{teor}

Per ottenere uno schema di ordine superiore al primo
nel caso dell'equazione \eqref{eq:onde-lineare-1D} dobbiamo quindi
rinunciare a tutte le proprietà della catena \eqref{eq:catena}, oppure
all'utilizzo di uno schema numerico lineare.
Nel caso di una generica equazione non lineare, invece, vedremo
che sarà opportuno concentrarsi su metodi che non aumentano
la variazione totale.

\section{Appunti sul capitolo 5 di Hesthaven}

Ma perché stare a complicarsi la vita con il flusso di Lax-Friedrichs,
con tecniche di upwinding, o altro? Se in uno schema in forma di conservazione
\begin{equation} \label{eq:schema-forma-conservazione}
U_j^{n+1} = U_j^n - \frac{k}{h} (F_{j+1/2}^n - F_{j-1/2}^n)
\end{equation}
il flusso $F_{j+1/2}^n$ rappresenta un'approssimazione di
$f(u(x_{j+1/2}))$, perché non scegliere
\[
F_{j+1/2}^n = \frac{f(u(x_{j})) + f(u(x_{j+1}))}{2}
\qquad
F_{j-1/2}^n = \frac{f(u(x_{j-1})) + f(u(x_{j}))}{2}
\]
e festa finita? Il problema è che si ottiene uno schema numerico consistente,
sì, ma instabile. Dimostriamo questo fatto nel caso più semplice possibile,
ossia per l'equazione delle onde lineare monodimensionale
\[
\partial_t u + c \partial_x u = 0
\]
con coefficiente $c$ costante.
Lo schema numerico è
\[
U_j^{n+1} = U_j^n - \frac{ck}{2h} U_{j+1}^n + \frac{ck}{2h} U_{j-1}^n
\]
Supponiamo che le condizioni al bordo siano periodiche,
cioè che gli indici $j$ abbiano valore modulo $N$.
Allora, indicando con $\hat{U}_k$ la trasformata di Fourier discreta
del vettore $U_j$, si ha
\begin{align*}
\hat{U}^{n+1}_k
& = \frac{1}{\sqrt{N}} \sum_{j=0}^{N-1} U^{n+1}_j e^{-i \frac{2\pi}{N} kj}
  = \frac{1}{\sqrt{N}} \sum_{j=0}^{N-1} \left(
	U_j^n - \frac{ck}{2h} U_{j+1}^n + \frac{ck}{2h} U_{j-1}^n
	\right) e^{-i \frac{2\pi}{N} kj} \\
& = \hat{U}^n_k
  - \frac{ck}{2h} e^{i \frac{2\pi}{N} k} \hat{U}^n_k
  + \frac{ck}{2h} e^{-i \frac{2\pi}{N} k} \hat{U}^n_k
  = \left(1 - i \frac{ck}{h} \frac{e^{i \frac{2\pi}{N} k}
                                  -e^{-i \frac{2\pi}{N} k}}{2i}
    \right) \hat{U}^n_k \\
& = \left(1 - i \frac{ck}{h} \sin(\frac{2\pi}{N} k)
    \right) \hat{U}^n_k
\end{align*}
Osserviamo che per quasi ogni valore di $k$ il coefficiente
\[
1 - i \frac{ck}{h} \sin(\frac{2\pi}{N} k)
\]
ha modulo maggiore di 1, quindi il modulo di $\hat{U}^n_k$ tende
a infinito per $n \to \infty$. Di conseguenza, tende a infinito anche la norma
euclidea di $\hat{U}^n$, che per per il teorema di Plancherel discreto
è uguale a quella di $U^n$. In conclusione, la soluzione numerica ottenuta
è incondizionatamente instabile.
Questo tipo di analisi è noto in letteratura come \emph{Von Neumann stability analysis}
e funziona per tutti gli schemi numerici lineari (cioè, basati su
stencil costanti) applicati a equazioni lineari.

Ripetendo questa analisi nel caso del flusso di Roe, si vede invece come
la tecnica di upwinding produca uno schema numerico stabile, a patto che
\[
\abs{c}k \leq h
\]
Questa disuguaglianza, nota come \emph{condizione CFL}, si generalizza
a qualunque sistema iperbolico di equazioni alle derivate parziali:

\begin{prop}
Una condizione necessaria (non sempre sufficiente) alla stabilità
di uno schema numerico per un sistema iperbolico di equazioni alle derivate parziali
è che a ogni passo il dominio di dipendenza numerica
includa il dominio di dipendenza analitica.
\end{prop}

\noindent Tutti gli schemi numerici visti finora hanno la parte temporale
dell'errore di troncamento che va a zero come $O(k)$. D'altra parte, visto
che sono schemi del primo ordine anche rispetto a $h$, non avrebbe senso
migliorare la tecnica di integrazione temporale, perché tanto
la condizione CFL pone una limitazione superiore a $k$ in termini di $h$.
Ad ogni modo, ci sono due approcci allo sviluppo di schemi numerici
di ordine superiore al primo rispetto al tempo.
Il primo si riconduce a un sistema di equazioni differenziali
ordinarie discretizzando il problema rispetto allo spazio
\[
\frac{d}{dt} U_j(t) = - \frac{F_{j+1/2}(t) - F_{j-1/2}(t)}{h}
\]
e poi integra numericamente il sistema di ODE con un metodo di ordine superiore
al primo (per esempio, un metodo di Runge-Kutta).
Il secondo comincia sviluppando in serie di Taylor la soluzione rispetto al tempo
\[
u(x,t^{n}+k)
= u(x,t^{n})
+ \partial_t u(x,t^n) k
+ \frac{1}{2} \partial_{tt} u(x,t^n) k^2
+ \dots
\]
e poi sfrutta la definizione della PDE
\[
\partial_t u + \diver(f(u)) = 0
\]
per sostituire tutte le derivate temporali con altrettante derivate spaziali
\begin{gather*}
\partial_t u(x,t^n) = -\diver(f(u(x,t^n))) \\
\partial_{tt} u(x,t^n) = \diver\Bigl( \partial_u f(u(x,t^n)) \diver(f(u(x,t^n))) \Bigr) \\
\partial_{ttt} u(x,t^n) = \dots
\end{gather*}
In questo modo, rimangono solo da valutare le derivate spaziali in
$u(x,t^n)$ per ottenere un'approssimazione di $u(x,t^{n}+k)$.
Questa tecnica, ispirata al teorema di Cauchy-Kovalevskaya, è nota in questo
contesto come procedimento di \emph{Lax-Wendroff}.
Se ci limitiamo a uno sviluppo al secondo ordine rispetto al tempo e discretizziamo
le derivate spaziali con differenze finite centrate, otteniamo uno schema
della forma \eqref{eq:schema-forma-conservazione} con flusso
\[
F(u,v) = \frac{f(u)+f(v)}{2}
       - \frac{k}{2} \partial_u f \Bigl( \frac{u+v}{2} \Bigr) \frac{f(v)-f(u)}{h}
\]
noto come \emph{flusso di Lax-Wendroff}. Lo schema numerico si può dimostrare
che, nonostante l'utilizzo di differenze finite centrate, è stabile
e ha ordine di convergenza quadratico. Tuttavia, lo schema presenta
forti oscillazioni in presenza di discontinuità.
Questo fatto lascia presagire che nello studio dei metodi di ordine
superiore al primo avranno grande importanza le tecniche in grado
di ridurre il più possibile il numero e l'ampiezza di tali oscillazioni.

Concludiamo riportando un risultato di convergenza per schemi monotoni:
\begin{teor}
Uno schema monotono in forma di conservazione, il cui flusso è
lipschitziano e consistente, è convergente, e la soluzione a cui converge
è una soluzione di entropia. Riguardo alla velocità di convergenza,
se $u_0$ ha variazione limitata, allora
\[
\norm{ u(t^n) - u^n }_{h,1} \leq C(t^n) \sqrt{h}
\]
\end{teor}

% time step size: paragrafo 6.3.2 toro
%  + remark 6.9 paragrafo 6.4
%  + rejection method?
% boundary conditions: 6.3.3 toro



























