\chapter{Equazioni di Eulero} \label{ch:euler-equations}

Questa è una prima stesura del capitolo sulle equazioni di Eulero.
L'introduzione e il paragrafo 1.3 sono un po' da rivedere,
anche in funzione di quello che verrà scritto in futuro,
mentre il resto mi sembra abbastanza completo.
\vspace{1em}

%	La fluidodinamica moderna, 
%	
%	nacque alle ...
%	con l'introduzione da parte di Eulero delle equazioni
%	che ancora oggi portano il suo nome.
%	Le equazioni di Eulero, inizialmente formulate soltanto
%	per fluidi incomprimibili o gas barotropici,
%	furono poi estese al caso di 
%	
%	L?introduzione delle equazioni di Eulero alla fine del ...
%	segna l'inizio della fluidodinamica moderna,
%	basata su un approccio differenziale locale,
%	fondata sulla matematica delle equazioni alle derivate perziali.
%	
%	
%	
%	Ancora oggi le equazioni di Eulero mantengono grande importanza
%	sia nelle 
%	
%	Le equazioni di Eulero sono un sistema di equazioni
%	alle derivate parziali introdotto per la prima volta da Eulero
%	...
%	e in seguito generalizzato a qualunque flusso adiabatico e non viscoso,
%	cioè flusso in cui si possono trascurare le in assenza di scambio di calore
%	e di forze di attrito interne al fluido.

In fluidodinamica, le \emph{equazioni di Eulero per fluidi compressibili}
sono un sistema di equazioni alle derivate parziali
adatto a descrivere il moto di un fluido di densità variabile nelle ipotesi
semplificative di flusso non viscoso e adiabatico, vale a dire
in assenza di forze d'attrito interne al fluido o trasmissione di calore.
Rispetto ad altri modelli fluidodinamici più completi,
% vedi paragrafo 2.3.2 di "un approccio libero ecc"
% e la pagina wikipedia sull'inviscid flow.
come le equazioni di Navier-Stokes, queste ipotesi rendono
le equazioni di Eulero totalmente prive di termini diffusivi,
cosicché la loro dinamica è governata interamente da termini del primo ordine
di tipo convettivo non lineare. Quest'ultimi
conferiscono alle equazioni di Eulero un carattere puramente iperbolico
e permettono la formazione e la propagazione di discontinuità durante il moto.
(qui probabilmente commento sul fatto che la ricerca di metodi numerici in grado di
preservare al meglio tali discontinuità sia un argomento chiave
di questo lavoro \dots)

In letteratura esistono più versioni delle equazioni di Eulero.
A distinguerle sono diversi aspetti:
il numero di dimensioni spaziali lungo le quali avviene il moto,
le particolari grandezze fisiche associate alle incognite dell'equazione,
le proprietà termodinamiche del fluido
e il tipo di interazioni con l'ambiente esterno.
In questo lavoro ci occuperemo nello specifico delle
equazioni di Eulero per gas perfetti in $\R^3$ in assenza di forze esterne.
Sotto tali ipotesi, le equazioni di Eulero formano un sistema di
cinque equazioni scalari che può essere scritto come
\emph{sistema iperbolico di leggi di conservazione}
\begin{equation} \label{eq:sistema-iperbolico-di-leggi-di-conservazione}
\partial_t u + \diver(F(u)) = 0,
\end{equation}
la cui particolare forma esprime in modo esplicito
la conservazione di quantità fisiche rilevanti,
quali la massa, la quantità di moto o l'energia.
È senz'altro notevole che la conservazione di tali quantità
non sia solo condizione necessaria alla scrittura delle equazioni
di Eulero (perché è noto che massa, quantità di moto
ed energia si conservano in un sistema isolato),
ma sia anche condizione sufficiente: queste leggi
di conservazione vincolano a tal punto il moto del fluido
da determinarlo univocamente, perché ne esauriscono tutti i gradi
di libertà.
Pertanto, quando più avanti esprimeremo le leggi di conservazione
in forma differenziale, otterremo un sistema \emph{chiuso} di equazioni
alle derivate parziali.

Il primo paragrafo di questo capitolo è dedicato a delle nozioni di base
di meccanica dei continui e di termodinamica.
Il secondo paragrafo è dedicato all'espressione in forma differenziale
delle leggi di conservazione della massa, della quantità di moto e
dell'energia, le quali ci permetteranno di scrivere le equazioni di Eulero
nella forma \eqref{eq:sistema-iperbolico-di-leggi-di-conservazione}.
Il terzo paragrafo è in lavorazione \dots

\section{Elements of continuum mechanics and thermodynamics}

\subsection*{Kinematics of a continuum}

The whole field of continuum mechanics is built on the fundamental idea that
the motion and the deformations of a domain $\Omega$ in $\R^n$
can be described by applying the principles of mechanics (for our purposes,
\emph{classical} mechanics) to each of its infinitesimal portions of matter.
Suppose that in the first instant of motion $t_0$ the continuum is
in a reference configuration $\Omega_{t_0}$, and that starting from
$t_0$ each of its infinitesimal portions of matter, which from
now on we will refer to as \emph{particles}, is moving in $\R^n$
in a regular and reversible way along a curve. Then, it is well-defined
and regular (we shall need at least $C^2$ regularity)
the \emph{trajectory} function
\[
\varphi(\vec{X},t) \colon \Omega_{t_0} \times [t_0,+\infty) \to \R^n,
\]
which describes the position at time $t \geq t_0$ of the particle which
was in $\vec{X}$ at time $t_0$. We denote by $\Omega_t$ the portion
of space that the domain $\Omega$ occupies at time $t$,
i.e.\ the set $\varphi(\Omega_{t_0},t)$.
By the assumption that the motion of each particle must be reversible,
it is also well-defined and regular the \emph{inverse trajectory} function
\[
\psi(\vec{x},t) \colon \Omega_t \times [t_0,+\infty) \to \Omega_{t_0},
\]
which takes a particle passing through $\vec{x}$ at time $t$ back
to its starting position in $\Omega_{t_0}$.

Looking at the definitions of $\varphi$ and $\psi$, it is clear that
\begin{gather*}
\varphi(\psi(\vec{x},t),t) = \vec{x}
\quad \text{for each $\vec{x} \in \Omega_t$ and for each $t \geq t_0$,} \\
\psi(\varphi(\vec{X},t),t) = \vec{X}
\quad \text{for each $\vec{X} \in \Omega_{t_0}$ and for each $t \geq t_0$.}
\end{gather*}
By taking the derivative with respect to $t$ on both sides
of the second equation and then operating the change of variable
$\vec{X} = \psi(\vec{x},t)$, we get that
\begin{equation} \label{eq:phi-psi-funzione-inversa}
\psi_{\vec{x}}(\vec{x},t) \, \varphi_t(\psi(\vec{x},t),t) + \psi_t(\vec{x},t) = 0.
\end{equation}
This identity will be useful later. Let us now see how kinematic quantities
of the continuum can be expressed through the functions $\varphi$ and $\psi$.

We define the velocity $\vec{v}(\vec{x},t)$ of the continuum as the
velocity of the particle that passes through $\vec{x}$ at time $t$,
provided that such a particle exists:
\[
\vec{v} \colon \Omega_t \times [t_0,+\infty) \to \R^n
\qquad \vec{v}(\vec{x},t) = \varphi_t(\psi(\vec{x},t),t).
\]
Velocity is therefore the time derivative of the trajectory function $\varphi$
evaluated at $\vec{X} = \psi(\vec{x},t)$: all trajectories described by $\varphi$
start from $\Omega_{t_0}$, so we need to trace $\vec{x}$ back to the reference
configuration $\Omega_{t_0}$ first. This operation is known in the literature as
the conversion from \emph{Eulerian coordinates} to \emph{Lagrangian coordinates}.
In this work we have preferred to always keep this conversion explicit
for additional clarity, however it is a well-established convention in continuum
mechanics to do it implicitly by treating $\vec{X}$ as a function of $\vec{x}$
(or vice versa) at every fixed instant in time.

Similarly, we define the acceleration $\vec{a}(\vec{x},t)$ of the continuum as
\[
\vec{a} \colon \Omega_t \times [t_0,+\infty) \to \R^n
\qquad \vec{a}(\vec{x},t) = \varphi_{tt}(\psi(\vec{x},t),t).
\]
The function $\psi$ in the definitions of velocity and acceleration
makes it so that, unlike the motion of a point particle in an inertial
frame of reference, $\vec{a}$ is not just the time derivative of $\vec{v}$.
However, by the chain rule and identity \eqref{eq:phi-psi-funzione-inversa},
we can prove that
\begin{align} \label{eq:derivata-euleriana-u}
\vec{v}_t(\vec{x},t)
&= \varphi_{t\vec{x}}(\psi(\vec{x},t),t) \, \psi_t(\vec{x},t)
 + \varphi_{tt}(\psi(\vec{x},t),t) \nonumber \\
&= - \varphi_{t\vec{x}}(\psi(\vec{x},t),t) \, \psi_{\vec{x}}(\vec{x},t)
\, \varphi_t(\psi(\vec{x},t),t) + \vec{a}(\vec{x},t) \nonumber \\
&= -\vec{v}_{\vec{x}}(\vec{x},t) \vec{v}(\vec{x},t) + \vec{a}(\vec{x},t).
\end{align}
The term $\vec{v}_{\vec{x}}(\vec{x},t) \vec{v}(\vec{x},t)$ denotes the
product between the Jacobian matrix of $\vec{v}$ with respect to
its spatial variables and the vector $\vec{v}$ itself.
Crucially, this term is nonlinear.

Looking at identity~\eqref{eq:derivata-euleriana-u}, it feels natural
to define a new differential operator $D_t$, known as
\emph{Lagrangian derivative}, such that
\[
D_t \vec{v}(\vec{x},t) \deq \vec{v}_t(\vec{x},t)
+ \vec{v}_{\vec{x}}(\vec{x},t) \vec{v}(\vec{x},t)
= \vec{a}(\vec{x},t).
\]
The fact that acceleration in a continuum can be expressed
as the Lagrangian derivative of velocity, without any
reference to $\varphi$ and $\psi$, is a simple yet remarkable result.
More generally, we can define the Lagrangian derivative of
any tensor field $\tns{g}(x,t)$ as
\begin{equation} \label{eq:derivata-lagrangiana}
D_t \tns{g}(\vec{x},t)
= \partial_t \tns{g}(\vec{x},t)
+ \partial_i \tns{g}(\vec{x},t) \, \vec{v}^i(\vec{x},t).
\end{equation}
From a physical point of view, the Lagrangian derivative $D_t$
can be understood as the variation of the quantity~$\tns{g}$ as
measured by an observer who is moving along the trajectory
of a particle of the continuum, whereas the usual time derivative $\partial_t$,
also known in this context as \emph{Eulerian derivative}, can be understood
as the variation of the quantity~$\tns{g}$ as measured by a stationary
observer in $\R^n$.
For a short introduction to tensors and their notation,
including the Einstein summation convention,
we refer to Appendix \ref{ch:appendix-tensors}.

\subsection*{Cauchy stress tensor}

The deformations of a continuum throughout its motion are caused by
the forces that each infinitesimal portion of matter exerts on
its surroundings. Therefore, the description of such forces is
a topic of utmost importance in continuum mechanics.
Let us fix an instant in time $t \geq t_0$, a point $\vec{x}$ belonging to
the continuum $\Omega_t$ in $\R^n$, and a versor $\vec{\nu} \in \R^n$.
Since $\vec{\nu}$ has norm 1, we can write $\vec{\nu} \in S^{n-1}$,
with $S^{n-1}$ being the unit sphere in $\R^n$.
Let $S$ be an oriented smooth hypersurface in $\Omega_t$ such that
$\vec{x}$ belongs to $S$ and the normal vector
to $S$ in $\vec{x}$ is equal to $\vec{\nu}$.
Furthermore, let $W_1$ and $W_2$ be two portions of continuum of positive
measure such that
\[
W_1 \subset \Omega_t,
\qquad W_2 \subset \Omega_t,
\qquad W_1 \cap W_2 = \emptyset,
\qquad \partial W_1 \cap \partial W_2 = S.
\]
Without loss of generality, we can suppose that the normal $\vec{\nu}$
is pointing away from $W_1$ and into $W_2$. We denote by $\vec{F}(W_1,W_2)$
the contact force that $W_2$ exerts on $W_1$ at time $t$ through the
separating hypersurface $S$. Let $A(R)$ be the surface measure
of any given hypersurface $R$, and let $B(\vec{x},h)$ be the open ball centered
at $\vec{x}$ with radius $h > 0$.
If for each $t \geq t_0$, $\vec{x} \in \Omega_t$, $\vec{\nu} \in S^{n-1}$
the limit
\begin{equation} \label{eq:limite-sforzi}
\tns{\tau} =
\lim_{h \to 0}
	\frac{\vec{F}\bigl(W_1 \cap B(\vec{x},h), \, W_2 \cap B(\vec{x},h)\bigr)}
	     {A\bigl(S \cap B(\vec{x},h)\bigr)}
\end{equation}
exists, is finite, and doesn't depend on the choice of $S$, $W_1$, $W_2$,
then the \emph{Cauchy stress vector}
\[
\tns{\tau}(\vec{x},t,\vec{\nu})
\colon \Omega_t \times [t_0,+\infty) \times S^{n-1} \to \R^n
\]
is well-defined.
The importance of this force density $\tns{\tau}$ is that,
once known, it can be integrated on any hypersurface $S$ in $\Omega_t$
to compute the macroscopic force of contact between two
portions of continuum separated by $S$.

Throughout its motion, a continuum is subject to two kinds of external forces:
volume forces and surface forces. An example of volume force is gravity,
which acts on all infinitesimal portions of the continuum;
an example of surface force is the one exerted by the walls
of a container on a fluid to prevent its expansion (like in a gas bottle).
As long as they are smooth functions of $\vec{x}$ and $t$,
external volume forces don't affect the stress tensor in any way,
because their contribution to limit \eqref{eq:limite-sforzi}
is an additional term in the numerator which goes to zero like $h^n$,
whereas the denominator goes to zero like $h^{n-1}$.
We can therefore ignore external volume forces for the time being.

External surface forces, on the other hand, allow us to define
the stress vector $\tns{\tau}$ on the boundary of $\Omega_t$.
Formally, let $\vec{x} \in \partial \Omega_t$ and let $\vec{\nu}(\vec{x})$
be the normal to $\partial \Omega_t$ in $\vec{x}$ pointing away from $\Omega_t$.
Then, if we denote the external surface force density in $\vec{x}$ at
time $t$ by $\vec{f}_s^{\text{ext}}(\vec{x},t)$, we get that
\begin{equation} \label{eq:sforzo-sul-bordo}
\tns{\tau}(\vec{x},t,\vec{\nu}(\vec{x}))
\deq \vec{f}_s^{\text{ext}}(\vec{x},t)
\quad \text{for all $\vec{x} \in \partial \Omega_t$}.
\end{equation}

\noindent The following theorem, due to Cauchy, describes the dependence
of $\tns{\tau}$ on the direction $\vec{\nu}$:
\begin{teor}[Cauchy’s stress theorem]
Suppose that the Cauchy stress vector is well-defined and sufficiently
smooth in each point of its domain. Then, there exists a type $(2,0)$
tensor field $\tns{\tau}^{ij}(\vec{x},t)$,
known as \emph{Cauchy stress tensor}, such that
\begin{enumerate}
\item $\tns{\tau}^{ij}(\vec{x},t) \vec{\nu}_j = \tns{\tau}(\vec{x},t,\vec{\nu})$
	for each $\vec{\nu} \in S^{n-1}$.
\item The tensor $\tns{\tau}$ is symmetric,
	i.e.\ $\tns{\tau}^{ij}(\vec{x},t) = \tns{\tau}^{ji}(\vec{x},t)$.
\end{enumerate}
\end{teor}

\noindent The proof of this theorem is classical, and can be found
in any introductory book on continuum mechanics.
The key insight is to apply Euler's first and second law of motion
(better known as \emph{equazioni cardinali della dinamica} in Italy)
to an infinitesimal portion of the continuum shaped like a tetrahedron.

\subsection*{Intensive and extensive properties of a continuum}

Physical properties of a continuum can often be categorized as being
either \emph{intensive} or \emph{extensive}. The former are local
properties that don't depend on the size of the continuum or on
the amount of matter that it contains, so they can be described
by some tensor field $\tns{g}(\vec{x},t)$. The variation of an intensive
property as measured along the trajectories of the particles of a continuum
is given by the Lagrangian derivative \eqref{eq:derivata-lagrangiana}.
Examples of intensive properties are mass density, pressure, velocity,
and temperature.

Extensive properties are global and additive, so their mathematical
description requires an integral over a portion of continuum
$W \subseteq \Omega$:
\begin{equation} \label{eq:definizione-estensiva}
\tns{G}(t) = \int_{W_t} \tns{g}(\vec{x},t) \dx.
\end{equation}
In this formula, $W_t$ denotes the subset of $\Omega_t$ given by
the particles of $W$ at time $t$. The tensor field $\tns{g}$ that defines
the extensive property $\tns{G}$ is known as its \emph{density}.
The time derivative of an extensive property is given by the following theorem:

\begin{teor}[Reynolds transport theorem] \label{teor:reynolds}
Let $W_t = \varphi(W_{t_0},t)$ be a sufficiently regular subset of $\R^n$.
Then,
\begin{align*}
\frac{d}{dt} \tns{G}(t)
&= \int_{W_t} D_t \tns{g} + \tns{g} \diver(\vec{v}) \dx
	& \text{(Lagrangian form)} \\
&= \int_{W_t} \partial_t \tns{g} + \diver_i(\tns{g} \otimes \vec{v}^i) \dx
	& \text{(Eulerian form)}
\end{align*}
% We omit the indices of the tensor field $\tns{g}$ for the sake of
% notational convenience.
% La notazione $\diver_i(g \otimes v^i)$ serve a rendere esplicito
% l'indice rispetto al quale viene calcolata la divergenza del tensore
% $g \otimes v$.
\end{teor}

\begin{proof}[Proof (sketch)]
First, we remove the time dependence from the domain of integration $W_t$
by switching to Lagrangian variables $\vec{X} = \psi(\vec{x},t)$.
Then, we differentiate under the integral sign and use Jacobi's formula
to evaluate the time derivative of the determinant of the Jacobian matrix
which is now part of the integrand.
Finally, we obtain the result in Lagrangian form by switching back to
the original variables. The result in Eulerian form follows easily
from the product rule:
\[
\diver_i(\tns{g} \otimes \vec{v}^i)
= \partial_i (\tns{g} \vec{v}^i)
= (\partial_i \tns{g}) \vec{v}^i + \tns{g} (\partial_i \vec{v}^i)
= D_t \tns{g} - \partial_t \tns{g} + \tns{g} \diver(\vec{v}). \qedhere
\]
\end{proof}
\noindent Examples of extensive properties are volume, mass, energy, and entropy.

\subsection*{Thermodynamics of perfect gases}

The classical thermodynamic state of a system is described
at every instant in time by a vector of $N$ macroscopic
physical quantities known as \emph{thermodynamic coordinates},
or \emph{state variables}.
Some of these quantities are intensive physical properties of the system,
whereas others are extensive properties.
In principle, the value of an intensive physical property doesn't have to
be constant throughout the whole system. This means that,
for a state variable to be globally well-defined, we need to restrict
our attention to quasistatic processes, that is, transformations
under which all state variables remain uniform throughout
the system at each instant of time.

For a macroscopic system, this uniformity assumption is only acceptable
if the timescale required by the system to reach thermodynamic equilibrium
after small perturbations is negligible compared to the timescale over which
the transformation happens. However, since the former timescale is proportional
to the size of the system, while the latter is size-invariant, the quasistatic
assumption is always valid for an infinitesimal portion of continuum.

Suppose that the thermodynamic system is made of a single chemical species
in a single phase. By \emph{Gibbs' phase rule}, such a system has only
two thermodynamic degrees of freedom: the values of all state variables
can be computed as soon as any two of them are known. To make this possible,
there must exist $N-2$ equations, known as \emph{equations of state},
that constrain the values that the $N$ thermodynamic coordinates can take.

Let us consider a thermodynamic system made of $n$ moles of \emph{perfect gas},
and fix pressure $p$, volume $V$, temperature $T$ and internal energy $U$
as state variables. A perfect gas is subject to the following equations of state:
\begin{equation} \label{eq:equazione-stato-gas-perfetti-pVT}
pV = nRT, \quad U = n C_v T.
\end{equation}
A formal definition of perfect gas would require microscopic models whose
introduction is beyond the scope of this work. Internal energy, for example,
is a concept that can only be fully understood in the context of
statistical mechanics, as it is defined as the sum of all microscopic
kinetic and potential energies of all molecules that make up the gas.
The constant $R$ is known as \emph{universal gas constant},
and its value is approximately \SI{8,3145}{\joule\per\mol\per\kelvin}.
The quantity $C_v$ is known as \emph{molar heat capacity at constant volume}
and is defined by the ratio
\[
C_v = \frac{1}{n} \left(\frac{\delta Q}{dT}\right)_{dV = 0}
\]
between the infinitesimal amount of heat $\delta Q$ absorbed by a gas
during any quasistatic transformation at constant volume
and the corresponding increase in temperature $dT$.
Likewise, molar heat capacity at constant \emph{pressure} is defined as
\[
C_p = \frac{1}{n} \left(\frac{\delta Q}{dT}\right)_{dp = 0.}
\]
In theory, $C_v$ and $C_p$ should be functions of the state variables,
like $(V,T)$ or $(p,T)$. However, for a perfect gas it is possible
to prove using microscopic models that $C_v$ and $C_p$ are constants
whose value depends only on the number of degrees of freedom $d$ associated
to the motion of the gas molecules:
\[
C_v = \frac{d}{2}R, \quad C_p = \frac{d+2}{2}R.
\]
In the case of monoatomic gases, $d = 3$, whereas for diatomic gases
(like air, approximately) we have $d = 5$. The identity $C_p = C_v + R$
is known as \emph{Mayer's relation}, and the ratio
\[
\gamma = \frac{C_p}{C_v} = \frac{d+2}{d} > 1
\]
is an important adimensional constant known as \emph{adiabatic index}.

Let $m$ denote the mass of the gas, $\rho$ its mass density $m/V$,
and $u$ its specific internal energy~$U/m$. By the two equations of
state \eqref{eq:equazione-stato-gas-perfetti-pVT} and Mayer's relation,
we can deduce that
\begin{equation} \label{eq:equazione-stato-gas-perfetti-p-rho-u}
\frac{pV}{m} = \frac{nRT}{m} \qquad
\frac{p}{\rho} = \frac{RU}{C_v m} \qquad
p = \frac{R}{C_v} \rho u = \frac{C_p - C_v}{C_v} \rho u = (\gamma-1) \rho u.
\end{equation}
This new equation of state, featuring $p$, $\rho$ and $u$ as state variables,
will be useful at the end of the next section to close
Euler's system of equations for the motion of a compressible fluid.

Another important state variable is \emph{entropy}. Suppose that a perfect
gas undergoes a transformation from an initial state identified by the
state variables $(p_i,\rho_i)$ to a final state identified by $(p_f,\rho_f)$.
The corresponding change in entropy is
\[
\Delta S = n C_v \log\left( \frac{p_f \rho_f^{-\gamma}}{p_i \rho_i^{-\gamma}} \right).
\]
%For our mathematical purposes, this formula will suffice to define
%entropy up to a constant that will be fixed on a case-by-case basis.
This formula is enough to define the entropy of a perfect gas up to a constant;
this is not the way entropy is usually defined in physics, as it is lacking
in both generality and motivation, however for our mathematical purposes
there is no need to make things more complicated (e.g.\ by
introducing statistical mechanics and the third law of thermodynamics).
The constant in the definition of entropy is fixed as soon as the value
of $S$ is assigned to a reference state (like air at a
standard condition for temperature and pressure).
Entropy is an extensive property of a thermodynamic system and
obeys the second law of thermodynamics: entropy of an isolated system
undergoing an arbitrary transformation is a non-decreasing function of time.

\section{Derivation of the compressible Euler equations}

As we have pointed out in the introduction to this chapter,
the term \emph{Euler equations} is slightly ambiguous: all equations
governing inviscid and adiabatic flow fall under this name,
regardless of the choice of physical variables for the unknowns,
of the equation of state, of the Eulerian/Lagrangian framework, etc.
In this chapter we shall derive the Euler equations pertaining
to perfect diatomic gases in $\R^n$ that are not subject to external
volume forces. In order to make it simpler to later reference all the
various working assumptions, we list them here once and for all:
\begin{description}
\item[(H1)] The flow is adiabatic: each portion of fluid exchanges
	no heat with the surrounding fluid while moving ($\delta Q = 0$).
\item[(H2)] The flow is inviscid: all contact forces in the interior
	of the fluid are due to pressure and not to internal friction
	between the layers of the fluid which are in relative motion.
	Therefore, the stress tensor has the form
	\[
	\tns{\tau}^{ij}(\vec{x},t) = -p(\vec{x},t) \tns{\delta}^{ij}.
	\]
	The choice of the Knonecker tensor $\tns{\delta}^{ij}$ is essentially forced
	by the fact that pressure gives rise to normal and isotropic stresses.
\item[(H3)] The thermodynamic coordinates of the fluid are constrained
	at every point and at all times by the equations of state
	of a perfect gas \eqref{eq:equazione-stato-gas-perfetti-pVT},
	or by their reformulation \eqref{eq:equazione-stato-gas-perfetti-p-rho-u}.
	The molecules of the gas are diatomic, so the adiabatic index
	$\gamma$ is equal to $7/5$.
\item[(H4)] There are no external volume forces.
\end{description}

\subsection*{Balance law of mass}

The mass $m$ of a portion of fluid $W \subseteq \Omega$ is an
extensive physical property that is conserved throughout the motion of the fluid:
\[
\frac{d}{dt} m(t) = 0.
\]
Its density, known as \emph{mass density}, is a scalar field
denoted by $\rho(\vec{x},t)$:
\[
m(t) = \int_{W_t} \rho(\vec{x},t) \dx.
\]
Throughout this work, we always assume that $\rho(\vec{x},t)$ is strictly
positive for all $\vec{x}$ in $\Omega_t$ and all $t \geq t_0$.
Without this assumption, we would have to allow vacuum states
and employ ad-hoc techniques to handle them in a satisfactory way.
In any case, fluid states with negative mass density, like those
produced by unstable numerical methods (see Section \dots), %TODO
are always unacceptable, as they have no physical meaning whatsoever.

By the Reynolds transport theorem in Eulerian form, it follows that
\begin{equation} \label{eq:continuita-forma-integrale}
0
= \frac{d}{dt} m(t)
= \frac{d}{dt} \int_{W_t} \rho \dx
= \int_{W_t} \partial_t \rho + \diver(\rho \vec{v}) \dx.
\end{equation}
This identity is known as \emph{continuity equation for mass in integral form}.
Since the domain $W_t$ is arbitrary in $\Omega_t$, the continuity of the
integrand implies a stronger, local version of the identity, known as
\emph{continuity equation for mass in differential form}:
\begin{equation} \label{eq:continuita-forma-differenziale}
\partial_t \rho + \diver(\rho \vec{v}) = 0
\quad \text{for each $(\vec{x},t) \in \Omega_t \times [t_0,+\infty)$.}
\end{equation}
The continuity equation for mass allows us to prove the following variant
of the Reynolds transport theorem:
\begin{teor}[Reynolds transport theorem,
	density per unit mass] \label{teor:reynolds-per-unita-di-massa}
If the extensive physical property $\tns{G}$ is defined as
\[
\tns{G}(t) = \int_{W_t} \rho(\vec{x},t) \tns{g}(\vec{x},t) \dx
\]
and $W_t = \varphi(W_{t_0},t)$ is a sufficiently regular
subset of $\R^n$, then
\[
\frac{d}{dt} \tns{G}(t) = \int_{W_t} \rho \, D_t \tns{g} \dx.
\]
\end{teor}

\noindent In the light of this theorem, it is often convenient
to choose the density of an extensive physical property
\emph{per unit mass}, that is, a density of the form $\rho \:\! \tns{g}$
instead of just $\tns{g}$ as in \eqref{eq:definizione-estensiva}.

\subsection*{Balance law of linear momentum}

The linear momentum $\vec{P}$ of a portion of fluid $W \subseteq \Omega$
is the following extensive property:
\[
\vec{P}(t) = \int_{W_t} \rho(\vec{x},t) \vec{v}(\vec{x},t) \dx.
\]
Euler's first law of motion states that the rate of change of
linear momentum in a system of masses is equal to the resultant
of all the external forces acting on the system:
\[
\frac{d}{dt} \vec{P}(t) = \vec{F}^{\text{ext}}(t).
\]
In the case of a continuum, it is useful to make a distinction
between two kinds of external forces: volume forces $\vec{F}_v^{\text{ext}}$
and surface forces $\vec{F}_s^{\text{ext}}$. External volume forces are
given by the integral of a volume force density per unit mass
$\vec{f}_v^{\text{ext}}$, whereas external surface forces are given
by the integral of a surface force density $\vec{f}_s^{\text{ext}}$:
\[
\vec{F}_v^{\text{ext}}(t)
= \int_{W_t} \rho(\vec{x},t) \vec{f}_v^{\text{ext}}(\vec{x},t) \dx
\qquad \vec{F}_s^{\text{ext}}(t)
= \int_{\partial W_t} \vec{f}_s^{\text{ext}}(\vec{x},t) \dsigma.
\]
We observe that the same surface force $\vec{F}_s$ can be
considered both external and internal, depending on the point of
view (that of $W_t$, or that of $\Omega_t$).
On the contrary, a volume force $\vec{F}_v$ that is external to
$W_t$ will always be external to $\Omega_t$ as well.
In our case, $\vec{F}_s^{\text{ext}}$ is by definition external
to~$W_t$, so equation \eqref{eq:sforzo-sul-bordo} allows us
to express $\vec{F}_s^{\text{ext}}$ through the stress tensor $\tns{\tau}$:
\[
\vec{F}_s^{\text{ext}}(t)
= \int_{\partial W_t} \tns{\tau}(\vec{x},t) \cdot \vec{\nu}(\vec{x}) \dsigma.
\]
As before, the function $\vec{\nu}(\vec{x})$ denotes the normal to
$\partial W_t$ in $\vec{x}$ pointing away from $W_t$.
It is not necessary to specify the index of $\tns{\tau}^{ij}$ against
which the scalar product is computed (the second one),
because the stress tensor is symmetric.
The four equations that we have seen so far, together with
Reynolds transport theorem and the divergence theorem, imply that
\begin{gather*}
\frac{d}{dt} \vec{P}(t) = \vec{F}_v^{\text{ext}}(t) + \vec{F}_s^{\text{ext}}(t) \\
\frac{d}{dt} \int_{W_t} \rho(\vec{x},t) \vec{v}(\vec{x},t) \dx
	= \int_{W_t} \rho(\vec{x},t) \vec{f}_v^{\text{ext}}(\vec{x},t) \dx
	+ \int_{\partial W_t} \tns{\tau}(\vec{x},t) \cdot \vec{\nu}(\vec{x}) \dsigma \\
\int_{W_t} \partial_t (\rho \vec{v}^i)
	+ \diver_j(\rho \vec{v}^i \otimes \vec{v}^j) \dx
	= \int_{W_t} \rho \, (\vec{f}_v^{\text{ext}})^i \dx
	+ \int_{W_t} \diver_j(\tns{\tau}^{ij}) \dx
	\quad \text{for each $i = 1,\dots,n$}.
\end{gather*}
Now, by assumptions \textbf{(H2)} and \textbf{(H4)} on the flow,
we get that
\[
\int_{W_t} \partial_t (\rho \vec{v}^i)
	+ \partial_j (\rho \vec{v}^i \vec{v}^j + p \tns{\delta}^{ij}) \dx
	= 0 \quad \text{for each $i = 1,\dots,n$}.
\]
As before, we can get a differential version of this identity
by exploiting the fact that $W_t$ is an arbitrary subset of $\Omega_t$
(continuity of the integrand remains a key technical assumption):
\begin{equation} \label{eq:bilancio-quantita-di-moto}
\partial_t (\rho v^i) + \partial_j (\rho v^i v^j + p\delta^{ij}) = 0.
\end{equation}

\subsection*{Balance law of energy}

The total energy $E$ of a portion of fluid $W \subseteq \Omega$
is given by the sum of three terms:
\begin{itemize}
\item The macroscopic kinetic energy $E_k$, whose
	density is $(1/2) \rho \!\!\: \norm{\vec{v}}^2$.
\item The macroscopic potential energy $E_p$, which in our case
	is always 0 by assumption \textbf{(H4)}.
\item The internal energy $U$, whose density per unit mass
	$u(x,t)$ is called \emph{specific internal energy}.
\end{itemize}
If we denote by $e(\vec{x},t)$ the density of total energy,
we can then write
\[
E(t)
= E_k(t) + E_p(t) + U(t)
= \int_{W_t} \frac{1}{2} \rho \norm{\vec{v}}^2 \dx
+ 0
+ \int_{W_t} \rho u \dx.
\]
Hence, the density of total energy $e$ is equal to
$(1/2) \rho \norm{\vec{v}}^2 + \rho u$.
The variation of total energy $E(t)$ in the time interval $[t,t+h]$
follows the first law of thermodynamics:
\[
E(t+h) - E(t) = Q + L.
\]
%We remark that the differential form of this law, namely
%$dE = \delta Q + \delta L$, cannot be used here because $W$
%is not subject to quasistatic transformations.
%Precisiamo che in questo caso non sarebbe lecito usare la versione
%differenziale di tale legge, cioè $dE = \delta Q + \delta L$,
%perché $W$ non subisce trasformazioni quasistatiche.
%Per aggirare l'ostacolo, abbiamo quindi considerato la variazione di
%$E$ sull'intervallo $[t,t+h]$.
The assumption of adiabatic flow \textbf{(H1)} assures us that
the heat $Q$ absorbed by $W$ in this time interval is zero.
We can then split the work $L$ done to $W$ by its surroundings
in two terms: one due to external volume forces, and the other due to
external surface forces:
\[
L = L_v + L_s
  = \int_{W_{[t,t+h]}} \hspace{-1em}
	\rho \, \vec{f}_v^{\text{ext}} \!\cdot \vec{v} \, \dx \hspace{-1.5pt} \dt \,
  + \int_{\partial W_{[t,t+h]}} \hspace{-1em}
  	\vec{f}_s^{\text{ext}} \!\cdot \vec{v} \, \dsigma \hspace{-2pt} \dt.
\]
On the one hand, it follows from the definition of time derivative,
assumption \textbf{(H4)} and the absolute continuity of the Lebesgue
integral that
\[
\frac{d}{dt} E(t)
	= \lim_{h \to 0} \frac{E(t+h)-E(t)}{h}
	= \lim_{h \to 0} \frac{Q+L}{h}
	= \lim_{h \to 0} \frac{1}{h} \int_{\partial W_{[t,t+h]}} \hspace{-1em}
  		\vec{f}_s^{\text{ext}} \!\cdot \vec{v} \, \dsigma \hspace{-2pt} \dt
	= \int_{\partial W_t} \vec{f}_s^{\text{ext}} \!\cdot \vec{v} \, \dsigma
\]
On the other hand, it follows from the definition of $e(\vec{x},t)$ and
from the Reynolds transport theorem that
\[
\frac{d}{dt} E(t)
	= \frac{d}{dt} \int_{W_t} e(\vec{x},t) \dx
	= \int_{W_t} \partial_t e + \diver(e \vec{v}) \dx
\]
Thus, if we combine these two equations, recall definition 
\eqref{eq:sforzo-sul-bordo} and apply the divergence theorem, we get that
\begin{gather*}
\int_{W_t} \partial_t e + \diver(e \vec{v}) \dx
	= \int_{\partial W_t} \vec{f}_s^{\text{ext}} \!\cdot \vec{v} \, \dsigma
	= \int_{\partial W_t} \tns{\tau}^{ij} \vec{v}_i \vec{\nu}_j \dsigma
	= \int_{W_t} \partial_j (\tns{\tau}^{ij} \vec{v}_i) \dx \\
0
	= \int_{W_t} \partial_t e + \diver(e \vec{v})
	- \diver_j(\tns{\tau}^{ij} \vec{v}_i) \dx
	= \int_{W_t} \partial_t e + \diver(e \vec{v})
	- \diver_j(-p \tns{\delta}^{ij} \vec{v}_i) \dx \\
\int_{W_t} \partial_t e + \diver(e \vec{v} + p \vec{v}) \dx = 0
\end{gather*}
This identity is in integral form; the corresponding differential form is
\begin{equation} \label{eq:bilancio-energia}
\partial_t e + \diver(e \vec{v} + p \vec{v}) = 0.
\end{equation}

\subsection*{Euler equations in conservation form}

We can now combine equations \eqref{eq:continuita-forma-differenziale},
\eqref{eq:bilancio-quantita-di-moto} and \eqref{eq:bilancio-energia}
in a single system of partial differential equations known as
\emph{compressible Euler equations in conservation form}:
\begin{equation} \label{eq:sistema-eulero}
\begin{cases}
\partial_t \rho + \diver(\rho \vec{v}) = 0 \\
\partial_t (\rho \vec{v}^i)
	+ \partial_j (\rho \vec{v}^i \vec{v}^j + p\tns{\delta}^{ij})
	= 0 \quad \text{for all $i = 1,\dots,n$} \\
\partial_t e + \diver(e \vec{v} + p \vec{v}) = 0.
\end{cases}
\end{equation}
This is a system of $n+2$ equations in $n+3$ variables, so we need
to express pressure as a function of $(\rho,\rho\vec{v},e)$ in order to close it.
This is possible thanks to hypotesis \textbf{(H3)}, according to which
we can apply the state equation for perfect gases
\eqref{eq:equazione-stato-gas-perfetti-p-rho-u} to every infinitesimal
portion of fluid:
\[
p
= (\gamma-1) \rho u
= (\gamma-1) \left( e - \frac{1}{2} \rho \norm{\vec{v}}^2 \right)
= (\gamma-1) \left( e - \frac{\norm{\rho \vec{v}}^2}{2\rho} \right)
\]
Operiamo ora un cambio di notazione, così da poter
scrivere il sistema \eqref{eq:sistema-eulero} in forma vettoriale e
con la notazione tipica delle equazioni alle derivate parziali.
La variabile $u$ non indicherà più l'energia interna specifica,
bensì il vettore delle incognite
\[
u(x,t) \colon \Omega_t \times [t_0,\infty) \to \R^5
\qquad u(x,t) = \begin{pmatrix} \rho(x,t) \\ \rho(x,t) v(x,t) \\ e(x,t) \end{pmatrix}
\]
e la variabile $F$ non indicherà più una generica forza macroscopica,
bensì la funzione \emph{flusso}
\[
F \colon \R^+ \! \times \R^3 \! \times \R \to \R^{5 \times 3}
\qquad
F(u) =
\frac{1}{u_1}
\begin{pmatrix*}[l]
u_1 u_2            & u_1 u_3            & u_1 u_4            \\
u_2 u_2 + u_1 p(u) & u_2 u_3            & u_2 u_4            \\
u_3 u_2            & u_3 u_3 + u_1 p(u) & u_3 u_4            \\
u_4 u_2            & u_4 u_3            & u_4 u_4 + u_1 p(u) \\
u_5 u_2 + u_2 p(u) & u_5 u_3 + u_3 p(u) & u_5 u_4 + u_4 p(u)
\end{pmatrix*}
\]
con
\begin{equation*}
p(u) = (\gamma-1) \left( u^5 - \frac{u_2^2+u_3^2+u_4^2}{2 u^1} \right)
\qquad \gamma = \frac{7}{5}.
\end{equation*}
A questo punto si verifica facilmente che il sistema \eqref{eq:sistema-eulero}
è equivalente a
\[
\partial_t u + \diver(F(u)) = 0.
\]
Per convenzione, la divergenza viene calcolata rispetto alle righe
della matrice $F(u)$.
Abbiamo quindi ottenuto le equazioni di Eulero sotto forma di
\emph{sistema di leggi di conservazione}.

%La funzione tensoriale \emph{flusso} $F \colon \R^5 \to \R^5 \otimes \R^3$,
%che per comodità riportiamo sotto forma di matrice (basta abbassare
%il secondo indice controvariante) e mettendo in evidenza $u_1^{-1}$:

\section{Proprietà delle equazioni di Eulero}

In questo capitolo provvisorio ho pensato di riassumere schematicamente
i risultati più importanti sulle equazioni di Eulero per gas perfetti
(da un punto di vista dell'analisi matematica).
Ho fatto una ricerca in letteratura per capire cosa si sappia
sulle loro soluzioni (molto poco, mi sono accorto).
I testi più utili per il caso multidimensionale (quello di interesse)
si sono rivelati~\cite{serre} e~\cite{benzoni-gavage-serre}.
Le cose da dire sarebbero molte, ma ancora non so quali saranno
utili successivamente per la trattazione numerica delle equazioni.
Inoltre, potrebbe essere meglio non introdurre qui
la teoria sui sistemi iperbolici di leggi di conservazione,
bensì in un paragrafo o un capitolo precedente.
Alcune idee su cosa includere in questo capitolo (forse troppe):

\subsection*{Teoria quasilineare}

\begin{itemize}
% 3.1 serre
\item Definizione di \emph{iperbolicità} di un sistema lineare
	a coefficienti costanti del primo ordine
	\[
	\partial_t u + \sum_{i=1}^n A_i \partial_i u = 0,
	\qquad A_i \in \R^{m \times m}
	\]
	Motivazione della definizione in quanto condizione necessaria e
	sufficiente affinché il problema di Cauchy
	sia ben posto in $L^2$ (oppure in $H^s$ spazio di Sobolev).
	Sia
	\[
	A(\xi) = \sum_{i=1}^n \xi_i A_i
	\quad 	\text{per ogni $\xi \in \R^n$.}
	\]
	Un sistema è iperbolico se e solo se gli autovalori di $A(\xi)$
	sono reali per ogni $\xi \in \R^n$ e la matrice $A(\xi)$
	è diagonalizzabile in modo uniforme.
	Interpretazione degli autovalori di $A(\xi)$ come velocità
	di propagazione delle onde piane nella direzione $\xi$.
\item I sistemi simmetrizzabili sono iperbolici. Definizione di sistema
	strettamente iperbolico (autovalori semplici) e costantemente iperbolico
	(autospazi di dimensione costante).
	I sistemi strettamente/costantemente iperbolici sono iperbolici.
% 3.2 serre
\item Definizione di \emph{iperbolicità} per il sistema quasilineare
	\[
	\partial_t u + \sum_{i=1}^n A_i(u) \partial_i u = 0,
	\qquad A_i \colon \R^n \to \R^{m \times m}
	\]
	La diagonalizzabilità di $A(\xi,u)$ è richiesta in modo uniforme
	anche rispetto a $u$.
	Stavolta l'iperbolicità non è condizione sufficiente all'esistenza
	di soluzioni del problema di Cauchy, servono ipotesi più forti
	(ad esempio, simmetrizzabilità). Le equazioni di Eulero
	formano un sistema quasilineare in ogni punto in cui
	la soluzione $u$ è regolare.
% 3.3 serre
\item Definizione di campo caratteristico come applicazione da $(\xi,u)$
	agli autovalori (detti velocità caratteristiche) e autospazi di $A(\xi,u)$.
	Definizione di campi caratteristici linearmente degeneri e
	genuinamente non lineari.
% 13.2.2 benzoni-gavage
\item Il sistema delle equazioni di Eulero è iperbolico, se $\rho > 0$.
	Calcolo dei campi caratteristici. Il sistema è strettamente iperbolico
	in 1D e costantemente iperbolico in 2D/3D.
	Formula per la velocità di propagazione delle onde
	(utile per la condizione numerica CFL in seguito).
% proposizione 1.4 benzoni-gavage o la prima prop. di sideris84
\item L'informazione si propaga dunque a velocità finita, limitata nella
	direzione $\xi$ dal massimo autovalore di $A(\xi,u)$.
	Definizione di dominio di dipendenza.
	Principio di località (prima proposizione di \cite{sideris84}).
	Un teorema analogo non può valere per le equazioni di Navier-Stokes
	a causa del carattere parabolico dei processi di diffusione.
% 13.2.3 benzoni-gavage
\item Il sistema delle equazioni di Eulero è simmetrizzabile, se $\rho > 0$.
	Questo risultato è alla base della teoria di esistenza locale di soluzioni
	classiche.
\end{itemize}

\subsection*{Soluzioni classiche del problema di Cauchy}

\begin{itemize}
\item Esistenza e unicità locale di soluzioni $C^1$ a partire da
	una condizione iniziale $H^s$ con $s > 5/2$.
	Ho trovato molti riferimenti, ognuno con un enunciato leggermente
	diverso: Teorema 3.6.1 in \cite{serre},
	Teorema 2.1 in \cite{chen-wang},
	Teorema 13.1 in \cite{benzoni-gavage-serre} e
	Teorema 5.1.1 in \cite{dafermos}.
	Quest'ultimo mi pare il più completo, perché è l'unico a occuparsi
	di regolarità rispetto al tempo maggiore di $C^1$ (inclusione 5.1.5)
	e a dimostrare anche la dipendenza continua dai dati iniziali
	(Teorema 5.2.1). Tuttavia, per poter applicare questo risultato
	alle equazioni di Eulero è richiesta una trasformazione affine,
	dato che il flusso $F$ è definito solo sull'insieme $\{\rho > 0\}$
	e invece la palla $\overline{\mathscr{B}_\rho}$ è centrata nell'origine.
	La modifica richiesta è spiegata bene nel commento che segue immediatamente
	il Teorema 10.1 in \cite{benzoni-gavage-serre}.
\item Sarebbe utile, al fine di giustificare l'uso di metodi numerici
	di ordine elevato, un risultato di regolarità $C^\infty$
	anziché $C^1$ (per dati iniziali $C^\infty$, ovviamente).
	Non ho trovato nessun riferimento in letteratura,
	perché le soluzioni classiche sono in genere definite come
	$C^1$ o localmente lipschitziane.
	Mi pare però che il risultato di regolarità rispetto al
	tempo menzionato nel punto precedente possa essere sfruttato
	per ottenere questo tipo di regolarità (la regolarità spaziale segue
	senza problemi dai teoremi di embedding di Sobolev).
\item L'esistenza globale di soluzioni classiche $C^\infty$, ancora aperta
	per Eulero incomprimibile o Navier-Stokes, qui è sicuramente falsa.
	Le equazioni di Eulero possono sviluppare singolarità in tempo finito
	anche in presenza di dati iniziali $C^\infty$.
	I teoremi precedenti ci dicono che esistono tre possibili motivi
	per cui una soluzione classica può smettere di esistere:
	\begin{itemize}
	\item $\rho \to 0$ in almeno un punto, cioè si viene a creare un vuoto nel fluido.
	\item $\norm{u}_\infty \to \infty$, cioè almeno una delle variabili del fluido
		sviluppa un asintoto verticale (fenomeno di blowup).
	\item $\norm{\nabla u}_\infty \to \infty$, cioè almeno una delle variabili
		del fluido sviluppa un flesso a tangente verticale (fenomeno di wave-breaking).
		Fenomeni di compressione che portano i fronti d'onda
		a diventare più ripidi al passare del tempo.
		Analogia con l'intersezione di caratteristiche per
		l'equazione di Burgers non viscosa $u_t + (u^2)_x = 0$.
	\end{itemize}
\item Ho trovato pochissimi risultati della forma: \emph{se la condizione iniziale
	$u_0$ soddisfa la proprietà X, allora la soluzione classica $u(t)$
	diventa singolare in tempo finito per il motivo Y}
	(un risultato è \cite{yin}, ma sono 40 pagine di articolo!).
	Comunque è fuori discussione che esistano dati $C^\infty$
	niente affatto patologici che sviluppano singolarità in tempo finito:
	in \cite{sideris85} l'autore dimostra che esistono condizioni iniziali (anche piccole)
	che in un tempo finito $T$ rendono singolare la quantità di moto in almeno un punto,
	nell'ipotesi che la soluzione $u(t)$ sia regolare fino a $T$.
	Purtroppo questa dimostrazione di perdita di regolarità
	non è costruttiva (il ragionamento è: supponiamo che la soluzione
	classica esista per un tempo maggiore di $T$,
	allora si verificherebbe un blowup, assurdo),
	quindi nulla vieta che si sviluppi prima di $T$ un altro tipo di singolarità,
	anzi, per come sono fatti i dati iniziali sembra più intuitivo
	che si formi un'onda di shock piuttosto che un blowup della quantità di moto.
	
\item C'è da dire che, rispetto al caso 1D, il processo di formazione delle singolarità
	non è ancora chiaro. Un'idea per dare esempi costruttivi di formazione
	di certi tipi di singolarità (per esempio, onde di shock)
	potrebbe essere quella di estendere esempi 1D al caso 3D in modo
	costante lungo i piani ortogonali all'asse $x$.
	In questo modo si potrebbe dedurre che tutte le singolarità che si possono
	formare in 1D si possono anche formare in 3D.
	Magari in 1D esistono esempi costruttivi ragionevolmente semplici.
	Non sono però sicuro al 100\%
	che l'estensione costante di una soluzione 1D sia una soluzione 3D,
	dovrei fare due conti\dots

\item Precisazione: in 3D i fenomeni di dispersione delle onde contrastano
	i fenomeni di compressione a tal punto da rallentare o addirittura
	impedire del tutto la formazione di singolarità.
	Questo fa sì che esistano condizioni di tipo dispersivo sui dati iniziali
	sufficienti a garantire l'esistenza di soluzioni classiche per tutti i tempi
	\cite{grassin}.
	Si può dimostrare che non possono esistere risultati analoghi in 1D
	(paragrafo 7.8 di \cite{dafermos}).
	Questa è quindi una differenza qualitativa importante tra la formazione
	di shock in 3D e in 1D.
	
\item Alla luce di tutto questo, è necessario introdurre il concetto di soluzioni deboli.
\end{itemize}

\subsection*{Soluzioni deboli del problema di Cauchy}

\begin{itemize}
\item Definizione di soluzione debole in $L^\infty$ del problema di Cauchy
	% paragrafo 3.5 serre
\item In 1D, è stata dimostrato che esiste $\delta > 0$ tale che
	il problema di Cauchy è ben posto
	(esistenza \textbf{globale}, unicità, dipendenza continua dai dati iniziali)
	se la condizione iniziale ha variazione totale minore di $\delta$,
	più ipotesi su campi caratteristici genuinamente non lineari
	oppure linearmente degeneri
	(esistenza globale dovuta a Glimm, il resto a Bressan et al.
	Un breve survey paper di Bressan è \cite{bressan}).
	Dunque in 1D lo spazio di funzioni giusto in cui lavorare è BV.
	Il problema per dati iniziali arbitrari è ancora aperto.
\item L'esistenza di soluzioni deboli in 3D è un problema totalmente aperto.
	Non è nemmeno chiaro in quale spazio di funzioni sia opportuno lavorare,
	quindi la scelta di $L^\infty$ nella definizione di soluzione
	debole non ha motivazioni profonde (mentre in altri contesti, più semplici,
	è la scelta giusta; comunque qui rimane un'ipotesi fisica importante).
\item La condizione di Rankine-Hugoniot è necessaria e sufficiente affinché
	una soluzione regolare che presenta una superficie di discontinuità
	sia una soluzione debole.
\item Velocità di propagazione di una discontinuità.
	Il caso delle equazioni di Eulero.
% 13.4 bensoni-gavage
\item Differenza tra una discontinuità dinamica (es.\ onda d'urto)
	e una discontinuità di contatto (es.\ strati di un vortice).
	Differenza tra una discontinuità dinamica compressiva e una espansiva
	(es.\ onda di rarefazione).
\item Altro problema: le soluzioni deboli, ammesso che esistano,
	non sono in generale uniche.
	Idea: escludiamo quelle che violano il secondo principio della termodinamica.
\end{itemize}

\subsection*{Soluzioni deboli di entropia del problema di Cauchy}

\begin{itemize}
% paragrago 10.2 benzoni-gavage
\item Definizione di \emph{entropia matematica} e \emph{flusso di entropia}
	per un sistema di leggi di conservazione.
	Definizione di soluzione debole di entropia.
	Le soluzioni deboli di entropia non sono invertibili rispetto al tempo.
\item Nel caso dell'equazione di Eulero, l'entropia termodinamica
	è un'entropia matematica concava (quindi $-S$ è convessa; in
	matematica e in fisica si usano due convenzioni diverse).
	% unicità di tale funzione?
%\item L'esistenza di una funzione entropia strettamente convessa
%	implica la simmetrizzabilità di un sistema quasilineare.
\item Se un sistema di leggi di conservazione ammette un'entropia convessa,
	allora ogni sua soluzione debole	di entropia coincide con la sua
	soluzione classica per tutto il tempo in cui quest'ultima esiste.
\item Disuguaglianza necessaria e sufficiente affinché una soluzione regolare
	che presenta una superficie di discontinuità sia una soluzione debole di entropia.
% 13.4 bensoni-gavage
\item Non tutte le discontinuità sono fisicamente ammissibili.
	Condizione di entropia: il salto di entropia dev'essere positivo.
	Una discontinuità soddisfa la condizione di entropia se e solo se è compressiva.
	Definizione di shock come Rankine-Hugoniot più condizione di entropia.
% pagina 409 benzoni-gavage
\item Condizione di ammissibilità di Lax per gli shock.
	Shock sufficientemente piccoli soddisfano automaticamente questa
	condizione. La condizione di Lax permette di dimostrare
	risultati di stabilità degli shock (esistenza locale nel tempo
	di una soluzione regolare a tratti in 3D) % teorema 21.11 benzoni-gavage
	e risultati di unicità delle soluzioni deboli di entropia in 1D.
\item L'unicità delle soluzioni deboli entropiche in 3D è un problema aperto,
	ma ci sono buone ragioni per pensare che
	la risposta sia negativa alla luce dei risultati ottenuti
	per le equazioni di Eulero isoentropiche \cite{chiodaroli}.
	In letteratura sono quindi state proposte nozioni più forti
	di soluzioni deboli di entropia (per esempio, basate su
	limiti singolari delle equazioni di Navier-Stokes),
	ma ancora non esiste un criterio per estrarre una soluzione debole canonica.
\end{itemize}

\subsection*{Problema di Cauchy con condizioni al bordo}

\begin{itemize}
\item L'aggiunta di condizioni al bordo al problema di Cauchy
	non rende affatto più semplice la sua soluzione.
	Anzi, in prima battuta non è nemmeno chiaro quale tipo di condizioni al bordo
	possano essere imposte. Per questo gran parte della teoria
	per le equazioni di Eulero (e sistemi iperbolici
	di leggi di conservazione in generale) è stata sviluppata sul dominio $\R^n$.

\item La difficoltà nella prescrizione di condizioni al bordo
	è comune a tutte le PDE iperboliche. Per esempio,
	già nell'equazione del trasporto lineare scalare
	\[
	u_t + cu_x = 0
	\qquad x \in [a,b]
	\]
	è evidente il fatto che le condizioni al bordo (per esempio, di Dirichlet)
	non possono essere imposte in entrambi gli estremi contemporaneamente,
	e che la scelta dell'estremo di inflow è dettata dal segno di $c$.
	In dimensione più alta, bisogna andare a esaminare lo spettro
	delle matrici $A(\xi,u)$.

\item Allo stato attuale della teoria, non esiste un criterio
	per stabilire la buona posizione globale di un IBVP
	per le equazioni di Eulero in 3D.
	Esiste solamente una teoria locale per soluzioni regolari (teorema 5.6.1
	\cite{dafermos}), del tutto analoga al caso senza bordo.

\item Una classificazione esaustiva delle condizioni al bordo
	per le equazioni di Eulero si trova nel capitolo 14 di \cite{benzoni-gavage-serre}.
	Pacchetti software di riferimento come \texttt{clawpack} permettono
	di gestire in modo automatico solo tre tipi di condizioni al bordo
	(documentazione \url{https://www.clawpack.org/bc.html}):
	\begin{itemize}
	\item Pareti trasmettenti in uscita (outflow)
	\item Condizioni al bordo periodiche
	\item Pareti isolanti (slip walls)
	\end{itemize}
	Il testo di Hesthaven suppone quasi sempre che le condizioni al bordo siano
	periodiche, oppure che il tempo di simulazione sia inferiore a quello richiesto
	affinché l'informazione contenuta nella condizione iniziale raggiunga il bordo.
	Nei paragrafi 5.3.1 e 11.5.1 accenna agli altri casi.
\end{itemize}

\subsection*{Soluzioni di riferimento}

\begin{itemize}
\item In conclusione, non esiste una teoria sulla buona posizione delle
	equazioni di Eulero in 3D su cui fare affidamento per
	lo sviluppo di metodi numerici stabili e accurati.
	% Possiamo imparare molto dal confronto col caso 1D, ma
	% in 3D le dinamiche sono incomparabilmente più ricche
	Per valutare la qualità delle soluzioni numeriche ottenute
	possiamo al massimo fare un confronto con quelle fornite da altri metodi numerici
	ritenuti affidabili, oppure con soluzioni di riferimento in forma chiusa.
\item Sarebbe utile avere delle soluzioni di riferimento per la stima
	dell'errore dei metodi numerici.
	Ho trovato diversi esempi in 2D, ma niente in 3D\dots
	Se fosse possibile aggiungere forze esterne di volume, allora
	con queste si potrebbe forzare qualunque soluzione desiderata
	(ma sarebbe un sistema iperbolico di leggi di bilancio,
	e non più di conservazione).
\item Alcune configurazioni iniziali interessanti: onde d'urto piane,
	onde d'urto sferiche, problema di Riemann, instabilità di Kelvin-Helmholtz,
	double mach reflection, vortice isoentropico, scontro tra shock e vortice, \dots
\end{itemize}

%\item Problema di Riemann
%\item Non esistono invarianti di Riemann per le equazioni in $\R^3$
%https://www.theoretical-physics.net/dev/fluid-dynamics/euler.html
