\chapter{Equazioni di Eulero}

Questa è una prima stesura del capitolo sulle equazioni di Eulero.
L'introduzione e il paragrafo 1.3 sono un po' da rivedere,
anche in funzione di quello che verrà scritto in futuro,
mentre il resto mi sembra abbastanza completo.
\vspace{1em}

%	La fluidodinamica moderna, 
%	
%	nacque alle ...
%	con l'introduzione da parte di Eulero delle equazioni
%	che ancora oggi portano il suo nome.
%	Le equazioni di Eulero, inizialmente formulate soltanto
%	per fluidi incomprimibili o gas barotropici,
%	furono poi estese al caso di 
%	
%	L?introduzione delle equazioni di Eulero alla fine del ...
%	segna l'inizio della fluidodinamica moderna,
%	basata su un approccio differenziale locale,
%	fondata sulla matematica delle equazioni alle derivate perziali.
%	
%	
%	
%	Ancora oggi le equazioni di Eulero mantengono grande importanza
%	sia nelle 
%	
%	Le equazioni di Eulero sono un sistema di equazioni
%	alle derivate parziali introdotto per la prima volta da Eulero
%	...
%	e in seguito generalizzato a qualunque flusso adiabatico e non viscoso,
%	cioè flusso in cui si possono trascurare le in assenza di scambio di calore
%	e di forze di attrito interne al fluido.

In fluidodinamica, le \emph{equazioni di Eulero per fluidi compressibili}
sono un sistema di equazioni alle derivate parziali
adatto a descrivere il moto di un fluido di densità variabile nelle ipotesi
semplificative di flusso non viscoso e adiabatico, vale a dire
in assenza di forze d'attrito interne al fluido o trasmissione di calore.
Rispetto ad altri modelli fluidodinamici più completi,
% vedi paragrafo 2.3.2 di "un approccio libero ecc"
% e la pagina wikipedia sull'inviscid flow.
come le equazioni di Navier-Stokes, queste ipotesi rendono
le equazioni di Eulero totalmente prive di termini diffusivi,
cosicché la loro dinamica è governata interamente da termini del primo ordine
di tipo convettivo non lineare. Quest'ultimi
conferiscono alle equazioni di Eulero un carattere puramente iperbolico
e permettono la formazione e la propagazione di discontinuità durante il moto.
(qui probabilmente commento sul fatto che la ricerca di metodi numerici in grado di
preservare al meglio tali discontinuità sia un argomento chiave
di questo lavoro \dots)

In letteratura esistono più versioni delle equazioni di Eulero.
A distinguerle sono diversi aspetti:
il numero di dimensioni spaziali lungo le quali avviene il moto,
le particolari grandezze fisiche associate alle incognite dell'equazione,
le proprietà termodinamiche del fluido
e il tipo di interazioni con l'ambiente esterno.
In questo lavoro ci occuperemo nello specifico delle
equazioni di Eulero per gas perfetti in $\R^3$ in assenza di forze esterne.
Sotto tali ipotesi, le equazioni di Eulero formano un sistema di
cinque equazioni scalari che può essere scritto come
\emph{sistema iperbolico di leggi di conservazione}
\begin{equation} \label{eq:sistema-iperbolico-di-leggi-di-conservazione}
\partial_t u + \diver(F(u)) = 0,
\end{equation}
la cui particolare forma esprime in modo esplicito
la conservazione di quantità fisiche rilevanti,
quali la massa, la quantità di moto o l'energia.
È senz'altro notevole che la conservazione di tali quantità
non sia solo condizione necessaria alla scrittura delle equazioni
di Eulero (perché è noto che massa, quantità di moto
ed energia si conservano in un sistema isolato),
ma sia anche condizione sufficiente: queste leggi
di conservazione vincolano a tal punto il moto del fluido
da determinarlo univocamente, perché ne esauriscono tutti i gradi
di libertà.
Pertanto, quando più avanti esprimeremo le leggi di conservazione
in forma differenziale, otterremo un sistema \emph{chiuso} di equazioni
alle derivate parziali.

Il primo paragrafo di questo capitolo è dedicato a delle nozioni di base
di meccanica dei continui e di termodinamica.
Il secondo paragrafo è dedicato all'espressione in forma differenziale
delle leggi di conservazione della massa, della quantità di moto e
dell'energia, le quali ci permetteranno di scrivere le equazioni di Eulero
nella forma \eqref{eq:sistema-iperbolico-di-leggi-di-conservazione}.
Il terzo paragrafo è in lavorazione \dots

\section{Richiami di meccanica dei continui e termodinamica}

\subsection*{Cinematica dei continui}

L'idea alla base della meccanica dei continui è quella di
descrivere il moto e le deformazioni di un dominio
$\Omega$ in $\R^n$ andando ad applicare i principi della meccanica
(per i nostri scopi, classica) a ogni porzione infinitesima di materia
che lo costituisce.
Supponiamo che nell'istante iniziale del moto $t_0$ il continuo si trovi in
una configurazione di riferimento $\Omega_{t_0}$, e che
a partire da $t_0$ ogni sua porzione infinitesima,
che chiameremo d'ora in poi \emph{particella},
si muova in $\R^n$ lungo una curva in modo regolare e reversibile.
Allora è ben definita e regolare (sarà opportuno almeno $C^2$) la funzione \emph{traiettoria}
\[
\varphi(X,t) \colon \Omega_{t_0} \times [t_0,+\infty) \to \R^n
\]
che descrive la posizione all'istante $t$ della particella
che all'istante $t_0$ si trovava in $X$.
Indichiamo con $\Omega_t$ la porzione di spazio occupata dal dominio
$\Omega$ all'istante $t$, cioè l'insieme $\varphi(\Omega_{t_0},t)$.
Per l'ipotesi di reversibilità del moto, è ben definita e regolare anche la funzione
\emph{traiettoria inversa}
\[
\psi(x,t) \colon \Omega_t \times [t_0,+\infty) \to \Omega_{t_0},
\]
che riconduce alla propria posizione iniziale in $\Omega_{t_0}$
ogni particella passante per $x$ all'istante $t$ .
Per come sono state definite $\varphi$ e $\psi$, è chiaro che
\begin{gather*}
\varphi(\psi(x,t),t) = x
\quad \text{per ogni $x \in \Omega_t$ e per ogni $t \geq t_0$,} \\
\psi(\varphi(X,t),t) = X
\quad \text{per ogni $X \in \Omega_{t_0}$ e per ogni $t \geq t_0$.}
\end{gather*}
Derivando quest'ultima identità rispetto a $t$
e operando il cambio di variabile $X = \psi(x,t)$, si ottiene
\begin{equation} \label{eq:phi-psi-funzione-inversa}
\psi_x(x,t) \, \varphi_t(\psi(x,t),t) + \psi_t(x,t) = 0.
\end{equation}
%\psi_x(\varphi(X,t),t) \, \varphi_t(X,t) + \psi_t(\varphi(X,t),t) = 0
Questa relazione ci tornerà utile in seguito.
Vediamo ora come le grandezze cinematiche del continuo
possano essere espresse mediante le funzioni $\varphi$ e $\psi$.
Definiamo la velocità $v(x,t)$ del continuo come la velocità
della particella (se esiste) che all'istante $t$ passa per $x$:
\[
v \colon \Omega_t \times [t_0,+\infty) \to \R^n
\qquad v(x,t) = \varphi_t(\psi(x,t),t).
\]
La velocità è quindi la derivata rispetto al tempo della funzione traiettoria,
con l'unica accortezza di ricondurre prima $x$
a una posizione $X = \psi(x,t)$ nella configurazione di riferimento
(dato che le traiettorie descritte da $\varphi$ partono sempre da $\Omega_{t_0}$).
Questa conversione è nota in letteratura come passaggio dalle
coordinate \emph{euleriane} alle coordinate \emph{lagrangiane}.
In questo paragrafo abbiamo preferito per una questione di chiarezza
rendere sempre esplicita tale conversione, tuttavia è una consuetudine
ben affermata in meccanica dei continui quella di trattare $X$ come
funzione di $x$ per un istante di tempo fissato e viceversa.

In modo analogo alla velocità,
definiamo l'accelerazione $a(x,t)$ del continuo come
\[
a \colon \Omega_t \times [t_0,+\infty) \to \R^n
\qquad a(x,t) = \varphi_{tt}(\psi(x,t),t).
\]
La presenza di $\psi$ nelle definizioni di velocità e accelerazione fa sì che,
a differenza del moto di un punto materiale in un sistema di
riferimento inerziale, $a$ non sia semplicemente la derivata rispetto
al tempo di $v$. Tuttavia, grazie alla relazione \eqref{eq:phi-psi-funzione-inversa},
è possibile dimostrare che
\begin{equation} \label{eq:derivata-euleriana-u}
v_t(x,t) = -v_x(x,t) v(x,t) + a(x,t).
\end{equation}
La notazione $v_x(x,t) v(x,t)$ indica il prodotto tra la matrice jacobiana
di $v$ rispetto alle variabili spaziali e il vettore colonna $v$ stesso.
Osserviamo che tale termine non è lineare.

Alla luce della~\eqref{eq:derivata-euleriana-u},
risulta naturale definire un nuovo operatore differenziale
$D_t$, detto \emph{derivata lagrangiana}, in modo tale che
\[
D_t v(x,t) \deq v_t(x,t) + v_x(x,t) v(x,t) = a(x,t).
\]
È un fatto notevole che si possa esprimere
l'accelerazione del continuo solamente come derivata (lagrangiana)
della funzione velocità, senza più fare riferimento alle funzioni
traiettoria e traiettoria inversa.
Più in generale, definiamo la derivata lagrangiana per un qualunque
campo tensoriale $g(x,t)$ come
\begin{equation} \label{eq:derivata-lagrangiana}
D_t g(x,t) = \partial_t g(x,t) + v^i(x,t) \partial_i g(x,t).
\end{equation}
(riferimento a un'ipotetica appendice sul calcolo tensoriale?) \\
Da un punto di vista fisico, la derivata lagrangiana $D_t$ si può
interpretare come la variazione della quantità~$g$ misurata
da un osservatore che si muove lungo la traiettoria
di una particella del continuo, in contrapposizione alla derivata
parziale $\partial_t$, detta \emph{euleriana}, che si può interpretare
come la variazione nel tempo di~$g$ misurata da un osservatore
stazionario in $\R^n$.

\subsection*{Tensore degli sforzi in $\R^3$}

Le deformazioni di un sistema continuo durante il moto
sono causate dalle forze che ogni porzione infinitesima di materia
esercita sulla materia circostante.
Pertanto, la descrizione di tali forze è un argomento di primaria
importanza in meccanica dei continui.

Fissiamo un istante di tempo $t \geq t_0$, un punto $x$ all'interno
del continuo $\Omega_t$ in $\R^3$ e una direzione qualunque $\nu \in S^3$.
Sia $S$ una superficie regolare contenuta in $\Omega_t$,
passante per $x$ e avente vettore normale in $x$ uguale a $\nu$.
Siano, inoltre, $W_1$ e $W_2$ due porzioni di continuo di misura
positiva tali che
\[
W_1 \subset \Omega_t,
\qquad W_2 \subset \Omega_t,
\qquad W_1 \cap W_2 = \emptyset,
\qquad \partial W_1 \cap \partial W_2 = S.
\]
Supponiamo che la normale $\nu$ sia uscente da $W_1$ ed entrante in $W_2$.
Indichiamo con $F(W_1,W_2)$ la forza di contatto che $W_2$ esercita
su $W_1$ al tempo $t$ attraverso la superficie $S$ che li separa.
Indichiamo inoltre con $A(S)$ l'area di tale superficie
e con $B(x,h)$ la palla aperta di centro $x$ e raggio $h$.
Se per ogni $t \geq t_0$, $x \in \Omega_t$, $\nu \in S^3$
esiste finito il limite
\begin{equation} \label{eq:limite-sforzi}
\tau =
\lim_{h \to 0}
	\frac{F\bigl(W_1 \cap B(x,h), \, W_2 \cap B(x,h)\bigr)}
	     {A\bigl(S \cap B(x,h)\bigr)}
\end{equation}
e se il valore di tale limite non dipende né dalla scelta della superficie $S$
né dalla scelta delle porzioni di continuo $W_1$ e $W_2$, allora
è ben definito il \emph{vettore degli sforzi di Cauchy}
\[
\tau(x,t,\nu) \colon \Omega_t \times [t_0,+\infty) \times S^3 \to \R^3.
\]
L'importanza di questa densità superficiale di forza $\tau$
è che, una volta nota, può essere integrata su una qualunque
superficie $S$ interna a $\Omega_t$ per calcolare la forza macroscopica 
di contatto tra due porzioni di continuo separate da $S$.

Durante il moto, un continuo può essere soggetto sia a forze esterne
di volume (come la forza di gravità), sia a forze esterne
di superficie (come quella esercitata dalla parete di un contenitore).
Le forze esterne di volume non danno alcun contributo al limite
\eqref{eq:limite-sforzi}, perché tendono a zero come $h^3$ a fronte
di un denominatore che tende a zero come $h^2$.
Per il momento possiamo quindi ignorarle.
Le forze esterne di superficie, invece, vanno a definire $\tau$
sul bordo di $\Omega_t$. Formalmente, sia $x \in \partial \Omega_t$
e $\nu$ la normale uscente da $\Omega_t$ in $x$. Allora, se indichiamo
con $f_s^e(x,t)$ la densità superficiale di forza esterna
in $x$ al tempo $t$, si ha che
\begin{equation} \label{eq:sforzo-sul-bordo}
\tau(x,t,\nu) = f_s^e(x,t).
\end{equation}

%Il seguente teorema, dovuto a Cauchy, descrive la dipendenza di $\tau$ da $\nu$:
%\begin{teor}[Teorema degli sforzi di Cauchy]
%Supponiamo che il vettore degli sforzi di Cauchy esista e sia sufficientemente
%regolare. Allora
%\begin{enumerate}
%\item Per il terzo principio della dinamica,
%	$\tau(x,t,-\nu) = -\tau(x,t,\nu)$.
%\item Per la prima equazione cardinale della dinamica,
%	$\tau$ è lineare rispetto a $\nu$.
%\item Per la seconda equazione cardinale della dinamica,
%	l'applicazione lineare che manda $\nu$ in $\tau$ fissati $x$ e $t$ è simmetrica.
%\end{enumerate}
%\end{teor}

\noindent Il seguente teorema, dovuto a Cauchy, descrive la dipendenza
di $\tau$ da $\nu$:
\begin{teor}[Teorema degli sforzi di Cauchy]
Supponiamo che il vettore degli sforzi di Cauchy sia ben definito e
sufficientemente regolare in ogni punto del suo dominio.
Allora esiste un campo tensoriale controvariante $\tau^{ij}(x,t)$,
detto \emph{tensore degli sforzi di Cauchy}, tale che
\begin{enumerate}
\item $\tau^{ij}(x,t) \nu_j = \tau(x,t,\nu)$ per ogni $\nu \in S^3$.
\item Il tensore $\tau$ è simmetrico, cioè $\tau^{ij}(x,t) = \tau^{ji}(x,t)$.
\end{enumerate}
\end{teor}

\noindent La dimostrazione di questo teorema consiste nell'applicazione
delle due equazioni cardinali della dinamica a delle opportune
porzioni di materia a forma di tetraedro e parallelepipedo.

\subsection*{Proprietà intensive ed estensive di un sistema continuo}

Le proprietà fisiche di un sistema continuo si dividono in \emph{intensive}
ed \emph{estensive}. Le prime sono locali e non dipendono dalla dimensione
del sistema o dalla quantità di materia in esso contenuta, quindi
possono essere descritte da un campo tensoriale $g(x,t)$
e di conseguenza la loro variazione nel tempo,
misurata su ogni porzione infinitesima di $\Omega$ durante il moto,
è data dalla derivata lagrangiana \eqref{eq:derivata-lagrangiana}.
Alcuni esempi di proprietà intensive sono la densità di massa,
la pressione e la temperatura.

Le proprietà estensive, d'altro canto, sono globali e additive,
quindi la loro descrizione matematica prevede l'uso di un integrale
su una porzione $W$ di $\Omega$ di misura positiva:
\begin{equation} \label{eq:definizione-estensiva}
G(t) = \int_{W_t} g(x,t) \dx.
\end{equation}
In questa formula, $W_t$ indica il sottoinsieme di $\Omega_t$
occupato dalle particelle di $W$ all'istante di tempo $t$.
Il campo tensoriale $g$ è detto \emph{densità} della grandezza
estensiva $G$.
La variazione nel tempo di una grandezza estensiva misurata
su $W_t$ durante il moto è data dal seguente teorema:

\begin{teor}[Teorema del trasporto di Reynolds] \label{teor:reynolds}
Se $W_t = \varphi(W_{t_0},t)$ è un insieme di $\R^n$ sufficientemente
regolare, allora
\begin{align*}
\frac{d}{dt} G(t)
&= \int_{W_t} D_t g + g \diver(v) \dx & \text{(forma lagrangiana)} \\
&= \int_{W_t} \partial_t g + \diver_i(g \otimes v^i) \dx & \text{(forma euleriana)}
\end{align*}
La notazione $\diver_i(g \otimes v^i)$ serve a rendere esplicito
l'indice rispetto al quale viene calcolata la divergenza del tensore
$g \otimes v$. Per semplicità, lasciamo muti gli indici del tensore $g$.
% indica che la divergenza del tensore $g \otimes v$ è calcolata
% rispetto all'indice di $v$.
\end{teor}

\begin{proof}[Dimostrazione (traccia)]
Passiamo alle variabili lagrangiane $X = \psi(x,t)$, al fine di rimuovere
la dipendenza temporale dal dominio di integrazione $W_t$.
Dopodiché, deriviamo sotto segno di integrale e usiamo la formula di Jacobi
per calcolare la derivata del determinante jacobiano.
Infine, torniamo alle variabili originali per ottenere la tesi
in forma lagrangiana. La tesi in forma euleriana si ottiene
immediatamente da quest'ultima sfruttando l'identità tensoriale
\[
\diver_i(g \otimes v^i)
= \partial_i (g v^i)
= (\partial_i g) v^i + g (\partial_i v^i)
= D_t g - \partial_t g + g \diver(v). \qedhere
\]
\end{proof}

\noindent Alcuni esempi di grandezze estensive sono il volume, la massa e l'energia.

\subsection*{Termodinamica di un gas perfetto}

Lo stato termodinamico classico di un sistema 
è descritto in ogni istante da un vettore di~$N$
grandezze fisiche macroscopiche %direttamente osservabili
dette \emph{coordinate termodinamiche} o \emph{variabili di stato}.
Affiché il valore di tali coordinate sia globalmente definito in modo
univoco, supponiamo per il momento che tutte le grandezze fisiche
intensive (per esempio, la pressione) siano in ogni istante uniformi
all'interno del sistema.
Più avanti, quando andremo a descrivere la dinamica di un fluido,
tale ipotesi sarà soddisfatta localmente.
Supponiamo inoltre di avere a che fare con un sistema omogeneo,
cioè composto da una sola specie chimica e una sola fase.
Per la \emph{regola delle fasi di Gibbs}, un tale sistema ha solamente
due gradi di libertà termodinamici, il che significa che è sempre possibile
dedurre il valore di tutte le variabili di stato a partire solamente
dal valore di due di esse.
Affinché ciò sia possibile, dovranno necessariamente esistere $N-2$
equazioni, dette \emph{equazioni di stato}, che legano in modo
implicito e tipicamente non lineare tutte le $N$ variabili termodinamiche del sistema.

Consideriamo un sistema composto da $n$ moli di gas
e scegliamo come variabili di stato la pressione $p$,
il volume $V$, la temperatura $T$ e l'energia interna $U$.
%cioè l'energia contenuta nel gas sotto forma di energia microscopica
%sia cinetica che potenziale.
Per i nostri scopi, sarà più che sufficiente il modello di \emph{gas perfetto},
per il quale valgono le equazioni di stato 
\begin{equation} \label{eq:equazione-stato-gas-perfetti-pVT}
pV = nRT, \qquad U = n C_v T.
\end{equation}
Una definizione formale di gas perfetto richiederebbe l'introduzione
di modelli microscopici
%TODO
che esulano dallo scopo di questo lavoro.%
\footnote{L'energia interna, per esempio, è definita come la somma
delle energie cinetiche e potenziali (microscopiche)
di tutte le particelle del gas. Si rende quindi necessario un approccio
di tipo statistico.}
La costante $R$ è detta \emph{costante universale dei gas}
e vale circa \SI{8,3145}{\joule\per\mol\per\kelvin}.
La quantità $C_v$ è detta \emph{calore molare a volume costante}
ed è definita dal rapporto
\[
C_v = \frac{1}{n} \left(\frac{\delta Q}{dT}\right)_{dV = 0}
\]
tra la quantità infinitesima $\delta Q$ di calore assorbito dal gas
durante una qualsiasi%
\footnote{Purché la trasformazione sia \emph{quasistatica},
cioè preservi la definizione globale e univoca di tutte
le coordinate termodinamiche in ogni istante delle trasformazione.}
trasformazione a volume costante
e il suo incremento di temperatura $dT$.
In linea di principio, $C_v$ dovrebbe essere una funzione di $V$ e
di $T$, ma si può dimostrare che per un gas perfetto tale quantità
è costante e il suo valore dipende unicamente dal numero $d$ di
gradi di libertà associati al moto delle molecole del gas:
\[
C_v = \frac{d}{2}R.
\]
Nel caso di un gas monoatomico, $d = 3$, mentre per un gas
diatomico (come l'aria, approssimativamente) si ha $d = 5$.
Con lo stesso principio con cui abbiamo definito il calore molare
a volume costante, è possibile definire anche il calore
molare a \emph{pressione} costante:
\[
C_p = \frac{1}{n} \left(\frac{\delta Q}{dT}\right)_{dp = 0}
\]
Nel caso di un gas perfetto è possibile dimostrare la \emph{relazione di Mayer}
tra i due calori molari
\[
C_p = C_v + R,
\]
da cui segue
\[
C_p = \frac{d}{2}R + R = \frac{d+2}{2}R.
\]
Il rapporto
\[
\gamma = \frac{C_p}{C_v} = \frac{d+2}{d} > 1
\]
è una quantità adimensionale importante,
detta \emph{coefficiente di dilatazione adiabatica}.

Indichiamo ora con $m$ la massa del gas, con $\rho$ la sua densità
$m/V$ e con $u$ la sua energia interna specifica $U/m$.
Dalle due equazioni di stato \eqref{eq:equazione-stato-gas-perfetti-pVT}
e dalla relazione di Mayer possiamo dedurre che
\begin{equation} \label{eq:equazione-stato-gas-perfetti-p-rho-u}
\frac{pV}{m} = \frac{nRT}{m} \qquad
\frac{p}{\rho} = \frac{RU}{C_v m} \qquad
p = \frac{R}{C_v} \rho u = \frac{C_p - C_v}{C_v} \rho u = (\gamma-1) \rho u.
\end{equation}
Questa nuova equazione di stato, espessa mediante le coordinate termodinamiche
$p$, $\rho$ e $u$, sarà molto utile in seguito, quando andremo a scrivere
il sistema delle equazioni di Eulero per i fluidi.

% In questo modo, queste ultime funzioni non compariranno esplicitamente
% nelle equazioni del moto

% \footnote{Per semplicità, evitiamo il formalismo degli spazi affini
% e lavoriamo direttamente in R3xR+.}

% In assenza di forze esterne, tali equazioni sono autonome
% e quindi, senza perdita di generalità, possiamo
% eliminare da phi la dipendenza da t_0 supponendo d'ora in poi che t_0 = 0.

\section{Deduzione delle equazioni di Eulero}

Come abbiamo ricordato all'inizio del capitolo,
in letteratura esistono più versioni delle equazioni di Eulero:
in questo paragrafo andremo a ricavare quelle per gas perfetti
diatomici in $\R^3$ in assenza di forze esterne di volume.
Affinché sia più facile in seguito fare riferimento alle varie
ipotesi di lavoro, le elenchiamo qui in modo schematico una volta per tutte:
\begin{description}
\item[(H1)] Il flusso è adiabatico, cioè ogni porzione di fluido
	scambia durante il moto una quantità di calore $Q$ con il
	fluido circostante pari a zero.
\item[(H2)] Il flusso non è viscoso, cioè le uniche forze di contatto
	interne al fluido sono dovute alla pressione e non a eventuali
	attriti viscosi. Pertanto il tensore degli sforzi ha la forma
	\[
	\tau^{ij}(x,t) = -p(x,t)\delta^{ij}.
	\]
	La scelta del tensore di Knonecker $\delta^{ij}$ è sostanzialmente
	obbligata dal fatto che gli sforzi di pressione all'interno di un fluido
	sono normali e isotropi.
\item[(H3)] Le coordinate termodinamiche del fluido sono legate in ogni punto
	 e in ogni istante dalle equazioni di stato dei gas perfetti
	 \eqref{eq:equazione-stato-gas-perfetti-pVT}, o dalla loro
	 riformulazione \eqref{eq:equazione-stato-gas-perfetti-p-rho-u}.
	 Il gas è diatomico, quindi il coefficiente di dilatazione adiabatica
	 $\gamma$ vale $7/5$.
\item[(H4)] Non sono presenti forze esterne di volume.
\end{description}

\subsection*{Conservazione della massa}

La massa $m$ di una porzione di fluido $W \subseteq \Omega$ è una
grandezza estensiva che si conserva durante il moto di $W$:
\[
\frac{d}{dt} m(t) = 0.
\]
La sua densità, detta \emph{densità di massa}, è un campo scalare
e lo si indica con $\rho(x,t)$:
\[
m(t) = \int_{W_t} \rho(x,t) \dx.
\]
Affinché la densità di massa abbia significato fisico, è necessario
supporre che $\rho$ sia strettamente positiva in ogni punto di $\Omega_t$
e per ogni istante $t \geq t_0$.

%TODO
(commento sul fatto che questa ipotesi non è da dare affatto per scontata:
più avanti vedremo che $\rho \to 0$ è uno dei possibili
motivi per cui una soluzione classica delle equazioni di Eulero
può non esistere per tutti i tempi)

Per il teorema di trasporto di Reynolds e le due equazioni precedenti,
si ha che
\begin{equation} \label{eq:continuita-forma-integrale}
0
= \frac{d}{dt} m(t)
= \frac{d}{dt} \int_{W_t} \rho \dx
= \int_{W_t} \partial_t \rho + \diver(\rho v) \dx.
\end{equation}
Questa relazione è nota come \emph{equazione di continuità
per la massa in forma integrale}.
Per l'arbitrarietà del dominio $W_t$ in $\Omega_t$
e per la continuità della funzione integranda,
possiamo dedurne anche una versione locale, più forte,
detta \emph{equazione di continuità per la massa in forma differenziale}:
\begin{equation} \label{eq:continuita-forma-differenziale}
\partial_t \rho + \diver(\rho v) = 0
\quad \text{per ogni $(x,t) \in \Omega_t \times [t_0,+\infty)$.}
\end{equation}
L'equazione di continuità per la massa permette di
dimostrare la seguente variante del teorema del trasporto di Reynolds:
\begin{teor}[Teorema del trasporto di Reynolds, densità
per unità di massa] \label{teor:reynolds-per-unita-di-massa}
Se la grandezza estensiva $G$ è definita come
\[
G(t) = \int_{W_t} \rho(x,t) g(x,t) \dx
\]
e se $W_t = \varphi(W_{t_0},t)$ è un insieme di $\R^n$ sufficientemente
regolare, allora
\[
\frac{d}{dt} G(t) = \int_{W_t} \rho \, D_t g \dx.
\]
\end{teor}

\noindent Alla luce di questo teorema, risulta spesso vantaggioso
(al fine di semplificare i calcoli) scegliere la densità di
una grandezza estensiva \emph{per unità di massa}, cioè
una densità della forma $\rho g$ anziché soltanto $g$ come nella
\eqref{eq:definizione-estensiva}.

\subsection*{Bilancio della quantità di moto}

La quantità di moto $P$ di una porzione di fluido $W \subseteq \Omega$ è la
seguente grandezza estensiva:
\[
P(t) = \int_{W_t} \rho(x,t) v(x,t) \dx.
\]
La prima equazione cardinale della dinamica afferma che la variazione
della quantità di moto di un sistema di masse è uguale alla
risultante delle forze esterne che agiscono su di esso:
\[
\frac{d}{dt} P(t) = F^e(t).
\]
Nel caso di un continuo, è opportuno distinguere due tipi di forze
esterne: quelle di volume~$F_v^e$ e quelle di superficie~$F_s^e$.
Le forze esterne di volume sono date dall'integrale di 
una densità volumetrica di forza per unità di massa $f_v^e$,
mentre le forze esterne di superficie sono date dall'integrale
di una densità superficiale di forza $f_s^e$:
\[
F_v^e(t) = \int_{W_t} \rho(x,t) f_v^e(x,t) \dx
\qquad F_s^e(t) = \int_{\partial W_t} f_s^e(x,t) \dsigma.
\]
Osserviamo che una stessa forza di superficie $F_s$ può essere
considerata sia esterna che interna, a seconda del punto di vista
(quello di $W_t$ o quello di $\Omega_t$).
Viceversa, una forza di volume $F_v$ esterna rispetto a $W_t$
sarà sempre esterna anche rispetto a $\Omega_t$.
In questo caso, $F_s^e$ è per definizione esterna a $W_t$,
quindi l'equazione \eqref{eq:sforzo-sul-bordo} ci permette di
esprimere $F_s^e$ mediante il tensore degli sforzi~$\tau$:
\[
F_s^e(t) = \int_{\partial W_t} \tau(x,t) \cdot \nu(x) \dsigma.
\]
La funzione $\nu(x)$ indica il vettore normale uscente da $W_t$ in $x$.
Non è necessario specificare l'indice di $\tau^{ij}$ rispetto al quale
viene calcolato il prodotto scalare (sarebbe $j$), perché il tensore
degli sforzi è simmetrico. Le quattro equazioni viste fino a ora,
insieme al teorema del trasporto di Reynolds e al teorema della divergenza,
implicano complessivamente che
\begin{gather*}
\frac{d}{dt} \int_{W_t} \rho(x,t) v(x,t) \dx
	= \int_{W_t} \rho(x,t) f_v^e(x,t) \dx
	+ \int_{\partial W_t} \tau(x,t) \cdot \nu(x) \dsigma \\
\int_{W_t} \partial_t (\rho v) + \diver_j(\rho v \otimes v^j) \dx
	= \int_{W_t} \rho f_v^e \dx
	+ \int_{W_t} \diver_j(\tau^j) \dx.
\end{gather*}
A questo punto, per le ipotesi sul fluido \textbf{(H2)} e \textbf{(H4)},
concludiamo che
\[
\int_{W_t} \partial_t (\rho v^i)
	+ \partial_j (\rho v^i v^j + p\delta^{ij}) \dx
= 0.
\]
Infine, l'arbitrarietà del dominio $W_t$ in $\Omega_t$ e la continuità
della funzione integranda ci permettono anche in questo caso di ricavare 
la forma differenziale dell'equazione precedente:
\begin{equation} \label{eq:bilancio-quantita-di-moto}
\partial_t (\rho v^i) + \partial_j (\rho v^i v^j + p\delta^{ij}) = 0.
\end{equation}

\subsection*{Bilancio dell'energia}

L'energia totale $E$ di una porzione di fluido $W \subseteq \Omega$
è data dalla somma di tre termini:
\begin{itemize}
\item L'energia cinetica macroscopica $E_k$, la cui
	densità è $\frac{1}{2} \rho \!\!\: \norm{v}^2$.
\item L'energia potenziale macroscopica $E_p$, che nel nostro caso
	è nulla per l'ipotesi \textbf{(H4)}.
\item L'energia interna $U$, la cui densità è $\rho u$.
	La quantità $u(x,t)$ è detta \emph{energia interna specifica}.
\end{itemize}
Indichiamo con $e(x,t)$ la densità dell'energia totale $E(t)$.
Per quanto detto, possiamo scrivere
\[
E(t) = \int_{W_t} \frac{1}{2} \rho \norm{v}^2 \dx
	 + \int_{W_t} \rho u \dx,
\qquad e = \frac{1}{2} \rho \norm{v}^2 + \rho u.
\]
La variazione dell'energia totale $E(t)$ è descritta dalla prima legge
della termodinamica:
\[
E(t+h) - E(t) = Q + L.
\]
Precisiamo che in questo caso non sarebbe lecito usare la versione
differenziale di tale legge, cioè $dE = \delta Q + \delta L$,
perché $W$ non subisce trasformazioni quasistatiche.
Per aggirare l'ostacolo, abbiamo quindi considerato la variazione di
$E$ sull'intervallo $[t,t+h]$.

L'ipotesi di flusso adiabatico \textbf{(H1)} ci assicura che il calore $Q$
assorbito da $W$ in questo intervallo di tempo è nullo.
Possiamo poi suddividere il lavoro $L$ compiuto dall'ambiente
su $W$ in un termine dovuto alle forze esterne di volume e un
termine dovuto a quelle di superficie:
\[
L = \int_{W_{[t,t+h]}} \rho f_v^e \cdot v \,\dx\!\!\!\;\dt
  + \int_{\partial W_{[t,t+h]}} f_s^e \cdot v \,\dsigma\!\!\!\:\dt.
\]
Allora, passando al limite per $h \to 0$ e applicando il teorema del
trasporto di Reynolds, si ha
\begin{gather*}
\frac{d}{dt} E(t)
	= \frac{d}{dt} \int_{W_t} e \dx
	= \int_{\partial W_t} f_s^e \cdot v \dsigma \\
\int_{W_t} \partial_t e + \diver(ev) \dx
	= \int_{\partial W_t} \tau^{ij} \nu_j v_i \dsigma
	= \int_{W_t} \diver_j(\tau^{ij} v_i) \dx \\
\int_{W_t} \partial_t e + \diver(ev) - \diver_j(-p \delta^{ij} v_i) \dx = 0
	\qquad \int_{W_t} \partial_t e + \diver(ev+pv) \dx = 0
\end{gather*}
Ancora una volta, abbiamo fatto uso dell'ipotesi \textbf{(H4)},
dell'equazione \eqref{eq:sforzo-sul-bordo} e del teorema della divergenza.
La corrispondente forma differenziale è
\begin{equation} \label{eq:bilancio-energia}
\partial_t e + \diver(ev+pv) = 0.
\end{equation}

\subsection*{Equazioni di Eulero in forma conservativa}

Possiamo ora combinare le equazioni
\eqref{eq:continuita-forma-differenziale},
\eqref{eq:bilancio-quantita-di-moto} e \eqref{eq:bilancio-energia}
in un unico sistema di equazioni alle derivate parziali,
detto \emph{sistema delle equazioni di Eulero in forma conservativa}:
\begin{equation} \label{eq:sistema-eulero}
\begin{cases}
\partial_t \rho + \diver(\rho v) = 0 \\
\partial_t (\rho v^i) + \partial_j (\rho v^i v^j + p\delta^{ij}) = 0 \\
\partial_t e + \diver(ev+pv) = 0
\end{cases}
\end{equation}
Per chiudere questo sistema di cinque equazioni scalari nelle sei incognite
\[
\rho, \rho v^1, \rho v^2, \rho v^3, e, p
\]
è necessario esprimere la pressione in funzione delle altre cinque variabili.
Questo è possibile grazie all'ipotesi~\textbf{(H3)}, secondo cui possiamo
applicare l'equazione di stato dei gas perfetti
\eqref{eq:equazione-stato-gas-perfetti-p-rho-u}
a ogni porzione infinitesima di fluido:
\[
p
= (\gamma-1) \rho u
= (\gamma-1) \left( e - \frac{1}{2} \rho \norm{v}^2 \right)
= (\gamma-1) \left( e - \frac{\norm{\rho v}^2}{2\rho} \right)
\]
Operiamo ora un cambio di notazione, così da poter
scrivere il sistema \eqref{eq:sistema-eulero} in forma vettoriale e
con la notazione tipica delle equazioni alle derivate parziali.
La variabile $u$ non indicherà più l'energia interna specifica,
bensì il vettore delle incognite
\[
u(x,t) \colon \Omega_t \times [t_0,\infty) \to \R^5
\qquad u(x,t) = \begin{pmatrix} \rho(x,t) \\ \rho(x,t) v(x,t) \\ e(x,t) \end{pmatrix}
\]
e la variabile $F$ non indicherà più una generica forza macroscopica,
bensì la funzione \emph{flusso}
\[
F \colon \R^+ \! \times \R^3 \! \times \R \to \R^{5 \times 3}
\qquad
F(u) =
\frac{1}{u_1}
\begin{pmatrix*}[l]
u_1 u_2            & u_1 u_3            & u_1 u_4            \\
u_2 u_2 + u_1 p(u) & u_2 u_3            & u_2 u_4            \\
u_3 u_2            & u_3 u_3 + u_1 p(u) & u_3 u_4            \\
u_4 u_2            & u_4 u_3            & u_4 u_4 + u_1 p(u) \\
u_5 u_2 + u_2 p(u) & u_5 u_3 + u_3 p(u) & u_5 u_4 + u_4 p(u)
\end{pmatrix*}
\]
con
\begin{equation*}
p(u) = (\gamma-1) \left( u^5 - \frac{u_2^2+u_3^2+u_4^2}{2 u^1} \right)
\qquad \gamma = \frac{7}{5}.
\end{equation*}
A questo punto si verifica facilmente che il sistema \eqref{eq:sistema-eulero}
è equivalente a
\[
\partial_t u + \diver(F(u)) = 0.
\]
Per convenzione, la divergenza viene calcolata rispetto alle righe
della matrice $F(u)$.
Abbiamo quindi ottenuto le equazioni di Eulero sotto forma di
\emph{sistema di leggi di conservazione}.

%La funzione tensoriale \emph{flusso} $F \colon \R^5 \to \R^5 \otimes \R^3$,
%che per comodità riportiamo sotto forma di matrice (basta abbassare
%il secondo indice controvariante) e mettendo in evidenza $u_1^{-1}$:

\section{Proprietà delle equazioni di Eulero}

In questo capitolo provvisorio ho pensato di riassumere schematicamente
i risultati più importanti sulle equazioni di Eulero per gas perfetti
(da un punto di vista dell'analisi matematica).
Ho fatto una ricerca in letteratura per capire cosa si sappia
sulle loro soluzioni (molto poco, mi sono accorto).
I testi più utili per il caso multidimensionale (quello di interesse)
si sono rivelati~\cite{serre} e~\cite{benzoni-gavage-serre}.
Le cose da dire sarebbero molte, ma ancora non so quali saranno
utili successivamente per la trattazione numerica delle equazioni.
Inoltre, potrebbe essere meglio non introdurre qui
la teoria sui sistemi iperbolici di leggi di conservazione,
bensì in un paragrafo o un capitolo precedente.
Alcune idee su cosa includere in questo capitolo (forse troppe):

\subsection*{Teoria quasilineare}

\begin{itemize}
% 3.1 serre
\item Definizione di \emph{iperbolicità} di un sistema lineare
	a coefficienti costanti del primo ordine
	\[
	\partial_t u + \sum_{i=1}^n A_i \partial_i u = 0,
	\qquad A_i \in \R^{m \times m}
	\]
	Motivazione della definizione in quanto condizione necessaria e
	sufficiente affinché il problema di Cauchy
	sia ben posto in $L^2$ (oppure in $H^s$ spazio di Sobolev).
	Sia
	\[
	A(\xi) = \sum_{i=1}^n \xi_i A_i
	\quad 	\text{per ogni $\xi \in \R^n$.}
	\]
	Un sistema è iperbolico se e solo se gli autovalori di $A(\xi)$
	sono reali per ogni $\xi \in \R^n$ e la matrice $A(\xi)$
	è diagonalizzabile in modo uniforme.
	Interpretazione degli autovalori di $A(\xi)$ come velocità
	di propagazione delle onde piane nella direzione $\xi$.
\item I sistemi simmetrizzabili sono iperbolici. Definizione di sistema
	strettamente iperbolico (autovalori semplici) e costantemente iperbolico
	(autospazi di dimensione costante).
	I sistemi strettamente/costantemente iperbolici sono iperbolici.
% 3.2 serre
\item Definizione di \emph{iperbolicità} per il sistema quasilineare
	\[
	\partial_t u + \sum_{i=1}^n A_i(u) \partial_i u = 0,
	\qquad A_i \colon \R^n \to \R^{m \times m}
	\]
	La diagonalizzabilità di $A(\xi,u)$ è richiesta in modo uniforme
	anche rispetto a $u$.
	Stavolta l'iperbolicità non è condizione sufficiente all'esistenza
	di soluzioni del problema di Cauchy, servono ipotesi più forti
	(ad esempio, simmetrizzabilità). Le equazioni di Eulero
	formano un sistema quasilineare in ogni punto in cui
	la soluzione $u$ è regolare.
% 3.3 serre
\item Definizione di campo caratteristico come applicazione da $(\xi,u)$
	agli autovalori (detti velocità caratteristiche) e autospazi di $A(\xi,u)$.
	Definizione di campi caratteristici linearmente degeneri e
	genuinamente non lineari.
% 13.2.2 benzoni-gavage
\item Il sistema delle equazioni di Eulero è iperbolico, se $\rho > 0$.
	Calcolo dei campi caratteristici. Il sistema è strettamente iperbolico
	in 1D e costantemente iperbolico in 2D/3D.
	Formula per la velocità di propagazione delle onde
	(utile per la condizione numerica CFL in seguito).
% proposizione 1.4 benzoni-gavage o la prima prop. di sideris84
\item L'informazione si propaga dunque a velocità finita, limitata nella
	direzione $\xi$ dal massimo autovalore di $A(\xi,u)$.
	Definizione di dominio di dipendenza.
	Principio di località (prima proposizione di \cite{sideris84}).
	Un teorema analogo non può valere per le equazioni di Navier-Stokes
	a causa del carattere parabolico dei processi di diffusione.
% 13.2.3 benzoni-gavage
\item Il sistema delle equazioni di Eulero è simmetrizzabile, se $\rho > 0$.
	Questo risultato è alla base della teoria di esistenza locale di soluzioni
	classiche.
\end{itemize}

\subsection*{Soluzioni classiche del problema di Cauchy}

\begin{itemize}
\item Esistenza e unicità locale di soluzioni $C^1$ a partire da
	una condizione iniziale $H^s$ con $s > 5/2$.
	Ho trovato molti riferimenti, ognuno con un enunciato leggermente
	diverso: Teorema 3.6.1 in \cite{serre},
	Teorema 2.1 in \cite{chen-wang},
	Teorema 13.1 in \cite{benzoni-gavage-serre} e
	Teorema 5.1.1 in \cite{dafermos}.
	Quest'ultimo mi pare il più completo, perché è l'unico a occuparsi
	di regolarità rispetto al tempo maggiore di $C^1$ (inclusione 5.1.5)
	e a dimostrare anche la dipendenza continua dai dati iniziali
	(Teorema 5.2.1). Tuttavia, per poter applicare questo risultato
	alle equazioni di Eulero è richiesta una trasformazione affine,
	dato che il flusso $F$ è definito solo sull'insieme $\{\rho > 0\}$
	e invece la palla $\overline{\mathscr{B}_\rho}$ è centrata nell'origine.
	La modifica richiesta è spiegata bene nel commento che segue immediatamente
	il Teorema 10.1 in \cite{benzoni-gavage-serre}.
\item Sarebbe utile, al fine di giustificare l'uso di metodi numerici
	di ordine elevato, un risultato di regolarità $C^\infty$
	anziché $C^1$ (per dati iniziali $C^\infty$, ovviamente).
	Non ho trovato nessun riferimento in letteratura,
	perché le soluzioni classiche sono in genere definite come
	$C^1$ o localmente lipschitziane.
	Mi pare però che il risultato di regolarità rispetto al
	tempo menzionato nel punto precedente possa essere sfruttato
	per ottenere questo tipo di regolarità (la regolarità spaziale segue
	senza problemi dai teoremi di embedding di Sobolev).
\item L'esistenza globale di soluzioni classiche $C^\infty$, ancora aperta
	per Eulero incomprimibile o Navier-Stokes, qui è sicuramente falsa.
	Le equazioni di Eulero possono sviluppare singolarità in tempo finito
	anche in presenza di dati iniziali $C^\infty$.
	I teoremi precedenti ci dicono che esistono tre possibili motivi
	per cui una soluzione classica può smettere di esistere:
	\begin{itemize}
	\item $\rho \to 0$ in almeno un punto, cioè si viene a creare un vuoto nel fluido.
	\item $\norm{u}_\infty \to \infty$, cioè almeno una delle variabili del fluido
		sviluppa un asintoto verticale (fenomeno di blowup).
	\item $\norm{\nabla u}_\infty \to \infty$, cioè almeno una delle variabili
		del fluido sviluppa un flesso a tangente verticale (fenomeno di wave-breaking).
		Fenomeni di compressione che portano i fronti d'onda
		a diventare più ripidi al passare del tempo.
		Analogia con l'intersezione di caratteristiche per
		l'equazione di Burgers non viscosa $u_t + (u^2)_x = 0$.
	\end{itemize}
\item Ho trovato pochissimi risultati della forma: \emph{se la condizione iniziale
	$u_0$ soddisfa la proprietà X, allora la soluzione classica $u(t)$
	diventa singolare in tempo finito per il motivo Y}
	(un risultato è \cite{yin}, ma sono 40 pagine di articolo!).
	Comunque è fuori discussione che esistano dati $C^\infty$
	niente affatto patologici che sviluppano singolarità in tempo finito:
	in \cite{sideris85} l'autore dimostra che esistono condizioni iniziali (anche piccole)
	che in un tempo finito $T$ rendono singolare la quantità di moto in almeno un punto,
	nell'ipotesi che la soluzione $u(t)$ sia regolare fino a $T$.
	Purtroppo questa dimostrazione di perdita di regolarità
	non è costruttiva (il ragionamento è: supponiamo che la soluzione
	classica esista per un tempo maggiore di $T$,
	allora si verificherebbe un blowup, assurdo),
	quindi nulla vieta che si sviluppi prima di $T$ un altro tipo di singolarità,
	anzi, per come sono fatti i dati iniziali sembra più intuitivo
	che si formi un'onda di shock piuttosto che un blowup della quantità di moto.
	
\item C'è da dire che, rispetto al caso 1D, il processo di formazione delle singolarità
	non è ancora chiaro. Un'idea per dare esempi costruttivi di formazione
	di certi tipi di singolarità (per esempio, onde di shock)
	potrebbe essere quella di estendere esempi 1D al caso 3D in modo
	costante lungo i piani ortogonali all'asse $x$.
	In questo modo si potrebbe dedurre che tutte le singolarità che si possono
	formare in 1D si possono anche formare in 3D.
	Magari in 1D esistono esempi costruttivi ragionevolmente semplici.
	Non sono però sicuro al 100\%
	che l'estensione costante di una soluzione 1D sia una soluzione 3D,
	dovrei fare due conti\dots

\item Precisazione: in 3D i fenomeni di dispersione delle onde contrastano
	i fenomeni di compressione a tal punto da rallentare o addirittura
	impedire del tutto la formazione di singolarità.
	Questo fa sì che esistano condizioni di tipo dispersivo sui dati iniziali
	sufficienti a garantire l'esistenza di soluzioni classiche per tutti i tempi
	\cite{grassin}.
	Si può dimostrare che non possono esistere risultati analoghi in 1D
	(paragrafo 7.8 di \cite{dafermos}).
	Questa è quindi una differenza qualitativa importante tra la formazione
	di shock in 3D e in 1D.
	
\item Alla luce di tutto questo, è necessario introdurre il concetto di soluzioni deboli.
\end{itemize}

\subsection*{Soluzioni deboli del problema di Cauchy}

\begin{itemize}
\item Definizione di soluzione debole in $L^\infty$ del problema di Cauchy
	% paragrafo 3.5 serre
\item In 1D, è stata dimostrato che esiste $\delta > 0$ tale che
	il problema di Cauchy è ben posto
	(esistenza \textbf{globale}, unicità, dipendenza continua dai dati iniziali)
	se la condizione iniziale ha variazione totale minore di $\delta$,
	più ipotesi su campi caratteristici genuinamente non lineari
	oppure linearmente degeneri
	(esistenza globale dovuta a Glimm, il resto a Bressan et al.
	Un breve survey paper di Bressan è \cite{bressan}).
	Dunque in 1D lo spazio di funzioni giusto in cui lavorare è BV.
	Il problema per dati iniziali arbitrari è ancora aperto.
\item L'esistenza di soluzioni deboli in 3D è un problema totalmente aperto.
	Non è nemmeno chiaro in quale spazio di funzioni sia opportuno lavorare,
	quindi la scelta di $L^\infty$ nella definizione di soluzione
	debole non ha motivazioni profonde (mentre in altri contesti, più semplici,
	è la scelta giusta; comunque qui rimane un'ipotesi fisica importante).
\item La condizione di Rankine-Hugoniot è necessaria e sufficiente affinché
	una soluzione regolare che presenta una superficie di discontinuità
	sia una soluzione debole.
\item Velocità di propagazione di una discontinuità.
	Il caso delle equazioni di Eulero.
% 13.4 bensoni-gavage
\item Differenza tra una discontinuità dinamica (es.\ onda d'urto)
	e una discontinuità di contatto (es.\ strati di un vortice).
	Differenza tra una discontinuità dinamica compressiva e una espansiva
	(es.\ onda di rarefazione).
\item Altro problema: le soluzioni deboli, ammesso che esistano,
	non sono in generale uniche.
	Idea: escludiamo quelle che violano il secondo principio della termodinamica.
\end{itemize}

\subsection*{Soluzioni deboli di entropia del problema di Cauchy}

\begin{itemize}
% paragrago 10.2 benzoni-gavage
\item Definizione di \emph{entropia matematica} e \emph{flusso di entropia}
	per un sistema di leggi di conservazione.
	Definizione di soluzione debole di entropia.
	Le soluzioni deboli di entropia non sono invertibili rispetto al tempo.
\item Nel caso dell'equazione di Eulero, l'entropia termodinamica
	è un'entropia matematica concava (quindi $-S$ è convessa; in
	matematica e in fisica si usano due convenzioni diverse).
	% unicità di tale funzione?
%\item L'esistenza di una funzione entropia strettamente convessa
%	implica la simmetrizzabilità di un sistema quasilineare.
\item Se un sistema di leggi di conservazione ammette un'entropia convessa,
	allora ogni sua soluzione debole	di entropia coincide con la sua
	soluzione classica per tutto il tempo in cui quest'ultima esiste.
\item Disuguaglianza necessaria e sufficiente affinché una soluzione regolare
	che presenta una superficie di discontinuità sia una soluzione debole di entropia.
% 13.4 bensoni-gavage
\item Non tutte le discontinuità sono fisicamente ammissibili.
	Condizione di entropia: il salto di entropia dev'essere positivo.
	Una discontinuità soddisfa la condizione di entropia se e solo se è compressiva.
	Definizione di shock come Rankine-Hugoniot più condizione di entropia.
% pagina 409 benzoni-gavage
\item Condizione di ammissibilità di Lax per gli shock.
	Shock sufficientemente piccoli soddisfano automaticamente questa
	condizione. La condizione di Lax permette di dimostrare
	risultati di stabilità degli shock (esistenza locale nel tempo
	di una soluzione regolare a tratti in 3D) % teorema 21.11 benzoni-gavage
	e risultati di unicità delle soluzioni deboli di entropia in 1D.
\item L'unicità delle soluzioni deboli entropiche in 3D è un problema aperto,
	ma ci sono buone ragioni per pensare che
	la risposta sia negativa alla luce dei risultati ottenuti
	per le equazioni di Eulero isoentropiche \cite{chiodaroli}.
	In letteratura sono quindi state proposte nozioni più forti
	di soluzioni deboli di entropia (per esempio, basate su
	limiti singolari delle equazioni di Navier-Stokes),
	ma ancora non esiste un criterio per estrarre una soluzione debole canonica.
\end{itemize}

\subsection*{Problema di Cauchy con condizioni al bordo}

\begin{itemize}
\item L'aggiunta di condizioni al bordo al problema di Cauchy
	non rende affatto più semplice la sua soluzione.
	Anzi, in prima battuta non è nemmeno chiaro quale tipo di condizioni al bordo
	possano essere imposte. Per questo gran parte della teoria
	per le equazioni di Eulero (e sistemi iperbolici
	di leggi di conservazione in generale) è stata sviluppata sul dominio $\R^n$.

\item La difficoltà nella prescrizione di condizioni al bordo
	è comune a tutte le PDE iperboliche. Per esempio,
	già nell'equazione del trasporto lineare scalare
	\[
	u_t + cu_x = 0
	\qquad x \in [a,b]
	\]
	è evidente il fatto che le condizioni al bordo (per esempio, di Dirichlet)
	non possono essere imposte in entrambi gli estremi contemporaneamente,
	e che la scelta dell'estremo di inflow è dettata dal segno di $c$.
	In dimensione più alta, bisogna andare a esaminare lo spettro
	delle matrici $A(\xi,u)$.

\item Allo stato attuale della teoria, non esiste un criterio
	per stabilire la buona posizione globale di un IBVP
	per le equazioni di Eulero in 3D.
	Esiste solamente una teoria locale per soluzioni regolari (teorema 5.6.1
	\cite{dafermos}), del tutto analoga al caso senza bordo.

\item Una classificazione esaustiva delle condizioni al bordo
	per le equazioni di Eulero si trova nel capitolo 14 di \cite{benzoni-gavage-serre}.
	Pacchetti software di riferimento come \texttt{clawpack} permettono
	di gestire in modo automatico solo tre tipi di condizioni al bordo
	(documentazione \url{https://www.clawpack.org/bc.html}):
	\begin{itemize}
	\item Pareti trasmettenti in uscita (outflow)
	\item Condizioni al bordo periodiche
	\item Pareti isolanti (slip walls)
	\end{itemize}
	Il testo di Hesthaven suppone quasi sempre che le condizioni al bordo siano
	periodiche, oppure che il tempo di simulazione sia inferiore a quello richiesto
	affinché l'informazione contenuta nella condizione iniziale raggiunga il bordo.
	Nei paragrafi 5.3.1 e 11.5.1 accenna agli altri casi.
\end{itemize}

\subsection*{Soluzioni di riferimento}

\begin{itemize}
\item In conclusione, non esiste una teoria sulla buona posizione delle
	equazioni di Eulero in 3D su cui fare affidamento per
	lo sviluppo di metodi numerici stabili e accurati.
	% Possiamo imparare molto dal confronto col caso 1D, ma
	% in 3D le dinamiche sono incomparabilmente più ricche
	Per valutare la qualità delle soluzioni numeriche ottenute
	possiamo al massimo fare un confronto con quelle fornite da altri metodi numerici
	ritenuti affidabili, oppure con soluzioni di riferimento in forma chiusa.
\item Sarebbe utile avere delle soluzioni di riferimento per la stima
	dell'errore dei metodi numerici.
	Ho trovato diversi esempi in 2D, ma niente in 3D\dots
	Se fosse possibile aggiungere forze esterne di volume, allora
	con queste si potrebbe forzare qualunque soluzione desiderata
	(ma sarebbe un sistema iperbolico di leggi di bilancio,
	e non più di conservazione).
\item Alcune configurazioni iniziali interessanti: onde d'urto piane,
	onde d'urto sferiche, problema di Riemann, instabilità di Kelvin-Helmholtz,
	double mach reflection, vortice isoentropico, scontro tra shock e vortice, \dots
\end{itemize}

%\item Problema di Riemann
%\item Non esistono invarianti di Riemann per le equazioni in $\R^3$
%https://www.theoretical-physics.net/dev/fluid-dynamics/euler.html
