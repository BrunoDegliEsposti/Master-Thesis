%  ===== FVM =====
%  Definizione volumi finiti
%  Celle, facce, spigoli
%  Cell-centered vs node-centered
%  Meshful vs meshless
%  Schemi semidiscreti vs schemi totalmente discreti
%  Media aritmetica dei flussi -> instabile -> Metodo di Godunov
%  Significato e utilità dell'upwinding
%  Schema numerico conservativo, convergenza a soluzione debole (se converge)
%  L'esistenza di soluzioni deboli è aperta, quindi nessun teorema che garantisca stabilità
%  Condizione necessaria alla stabilità per metodi espliciti: condizione CFL (astratta)
%  Condizioni al bordo delicate
%  
%  ===== Metodo di Godunov =====
%  Godunov classico 1D
%  Godunov per griglie non strutturate in nD
%  Invarianza per rotazioni dell'equazione di Eulero
%  Problema di Riemann esteso
%  Riemann solver esatto vs approssimato
%  In ogni caso, è importante che i flussi numerici siano consistenti e lipschitziani
%  
%  ===== Metodi di ordine superiore =====
%  Ricostruzioni ENO/WENO
%  Quadratura di ordine adeguato sulle facce
%  Metodo per ODE
%    oppure
%  Metodo di Lax-Wendorff
%    oppure
%  Una via di mezzo: metodo ADER (un passo, fully discrete, lax-wendroff procedure)
%  
%  ===== Condizioni al bordo =====
%  ===== Metodi runge-kutta SSTVD =====
%  ===== Flussi numerici HLL, HLLC, ROE, etc =====

\chapter{Metodo dei volumi finiti}

Il \emph{metodo dei volumi finiti} è una delle principali
tecniche di discretizzazione per la soluzione numerica di equazioni
alle derivate parziali in forma integrale.
L'idea alla base di questo metodo consiste nell'approssimare
la soluzione esatta dell'equazione con delle sue medie integrali,
calcolate su una partizione finita del dominio $\Omega \subseteq \R^n$.
Gli elementi di una tale partizione prendono il nome di \emph{celle},
mentre le intersezioni dei bordi di celle adiacenti sono dette \emph{facce}
(oppure \emph{spigoli}, nel caso in cui $\Omega \subseteq \R^2$).
Se l'equazione alle derivate parziali descrive un fenomeno di tipo
evolutivo, come nel caso dei sistemi iperbolici di leggi
di conservazione
\[
\partial_t u(x,t) + \diver(F(u(x,t))) = 0,
\qquad (x,t) \in \Omega \times [0,T],
\]
allora la dimensione temporale $[0,T]$ viene tipicamente esclusa
da tale discretizzazione e trattata in altro modo,

per esempio con il metodo delle linee.

viene tipicamente esclusa dal processo di discretizzazione appena descritto
e trattata in modo indipendente.
Se il metodo dei volumi finiti si limita a trasformare
l'equazione alle derivate parziali in un'equazione differenziale
ordinaria, l'approccio prende il nome di \emph{metodo delle linee},
oppure \emph{schema semi-discreto}.
...tecniche standard per ODE, ad esempio con un metodo di Runge-Kutta.
In alternativa, 

Nel caso del \emph{metodo delle linee}, 
viene tr


In ogni caso, il punto di arrivo è uno schema numerico
per aggiornare le medie integrali di $u$ a ogni passo temporale $\Delta T$.

\clearpage





La discretizzazione geometrica del dominio può essere più o meno rigida.
Discretizzazioni più rigide hanno il vantaggio di 
essere più semplici da costruire e da analizzare,
e permettono una maggiore ottimizzazione del codice.

Discretizzazioni più flessibili, invece, si adattano
meglio a domini irregolari, e possono essere più facilmente
raffinate in modo locale, come richiesto da schemi adattativi.

Un elevato grado di flessibilità permette addirittura
di muovere con facilità gli elementi della discretizzazione,
così da gestire in modo naturale deformazioni di $\Omega$,
oppure da catturare meglio fenomeni di trasporto
(per esempio, consentendo il passaggio a un punto di vista lagrangiano).

Il massimo della flessibilità è ovviamente raggiunto nel caso
in cui si arrivi a fare del tutto a meno di discretizzare
il dominio.

%meshless: celle = regioni di influenza di nodi sparsi nel dominio
%regioni di influenza disgiunte di forma politopica -> FVM classico node-centered
%regioni di influenza sovrapposte -> FVPM





riprodurre nel discreto tutte le proprietà qualitative che la
soluzione esatta possiede nel continuo.
Nel caso del sistema \dots, è immediato capire quale
sia una di queste proprietà: la conservazione
le incognite che compongono il vettore u sono variabili conservate.
Facendo riferimento alle equazioni di Eulero, questo significa
che 





Il caso più rigido è sicuramente quello delle griglie regolari,
che 






