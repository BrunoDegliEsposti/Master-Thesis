\chapter*{Introduction} \label{ch:introduction}
%\phantomsection
\addcontentsline{toc}{chapter}{Introduction}

Fast, accurate and reliable simulations of transport phenomena are of great
importance to scientific and engineering applications, but to this day remain
challenging despite decades of effort in the field of numerical analysis.

The reasons are manifold: first, the interplay between the numerical domain
of dependence at every point and the analytical domain of dependence is very
delicate and, if not handled correctly, can easily lead to the definition
of unstable numerical methods.
%(hence the need for upwinding, CFL conditions, etc).

Second, weak diffusion terms in the equations (or the lack thereof) allow
sharp features of the solution to propagate without substantial smoothing
effects. This a problem for low order numerical schemes, which suffer from
excessive dissipation, but also for high order schemes, which rely on regularity
assumptions that may only hold piecewise on the exact solution.
Moreover, in the inviscid nonlinear case, even smooth initial boundary
data can develop discontinuities in finite time due to wave-breaking phenomena.

And finally, a convergent numerical scheme can still produce unsatisfactory results
from a qualitative point of view, because it may not enforce exact
conservation at the discrete level, or it may produce spurious oscillations
around sharp features, or it may even converge to the wrong kind of weak
solution (for example, one that violates a suitable generalization of the
second law of thermodynamics).

In this work, we tackle some of these issues in the context of the
numerical solution of the 2D compressible Euler equations,
which are among the most important examples of hyperbolic systems of
conservation laws both from a practical and a historical point of view.
Our approach builds on well-established but modern techniques that
combine the finite volume method (FVM) with Godunov’s method and Weighted
Essentially Non-Oscillatory (WENO) reconstructions based on least-squares
polynomial approximation, following the works of Hu and Shu and others
\cite{hu1999weighted} \cite{kaser2005ader}.

After an introduction to the 2D compressible Euler equations
and the finite volume method, we define numerical schemes
of arbitrarily high order for the solution of hyperbolic systems
of conservation laws by using a least-squares polynomial approximation
technique. The resulting schemes are quantitatively accurate, but
produce unacceptable oscillations around discontinuities or other sharp
features of the solutions, so their use must be limited to the simulation
of smooth flows.

To fix this, we introduce a family of numerical schemes known as WENO that can
detect and avoid unwanted oscillations in the numerical solution.
More precisely, in Chapter \ref{ch:WENO} we successfully generalize
type-I WENO schemes from triangular meshes to arbitrary polygonal
meshes and show their effectiveness on Voronoi tessellations for
the solution of the 2D compressible Euler equations.
This is in line with a wider, modern trend towards greater flexibility in
the discretization of problems’ domains.

A significant amount of effort was put into the implementation
of these numerical methods, which are at times simple to describe by
words but actually very time-consuming to code.
The most relevant parts of the source code can be found in Appendix
\ref{ch:appendix-source-code}. The high-level routines of the finite
volume solver are written in MATLAB, whereas the low-level routines
are written in C++ for the sake of performance; the two parts
are connected through MATLAB's MEX interface.

Properly visualizing a numerical solution to a 2D partial differential equation
can be a challenging task, but is also one of the best ways to judge its quality
and gain confidence that the numerical methods have been implemented correctly.
For this reason, we have saved the results of our simulations in a format
(\code{.vtk}) that is compatible with Paraview, a popular open-source
data analysis and visualization application widely used in computational
fluid dynamics for post-processing tasks.

%This thesis is structured as follows.
%In Chapter \ref{ch:euler-equations}, the 2D compressible Euler equations
%are introduced and derived from \dots
%
%In Chapter \ref{ch:conservation-laws}, the Euler equations 
%%special case of hyp.sys.
%
%In Chapter \ref{ch:FVM}, the finite volume method is introduced
%and numerical schemes of arbitrarily high order for the solution
%of hyperbolic systems of conservation laws are defined
%using a least squares polynomial approximation technique.
%
%In Chapter \ref{ch:WENO}, we successfully generalize type-I WENO schemes
%from triangular meshes to arbitrary polygonal meshes and show their
%effectiveness on Voronoi tessellations for the solution of
%the 2D compressible Euler equations.
%This generalization is in line with a modern trend towards greater
%flexibility in the discretization of problems’ domains.














