\chapter*{Introduction} \label{ch:introduction}
%\phantomsection
\addcontentsline{toc}{chapter}{Introduction}

{\color{blue} Qui per ora ho fatto copia-incolla di una parte del progetto
di ricerca che avevo scritto per la domanda di dottorato.
Penso che possa essere una base di partenza ragionevole.}\\

Fast, accurate and reliable simulations of transport phenomena are of great
importance to scientific and engineering applications, but to this day remain
challenging despite decades of effort in the field of numerical analysis.

The reasons are manifold: first, the interplay between the numerical domain
of dependence at every point and the analytical domain of dependence is very
delicate and, if not handled correctly, can easily cause instability
(hence the need for upwinding, CFL conditions, etc).

Second, weak diffusion terms in the equations (or the lack thereof) allow
sharp features of the solution to propagate without substantial smoothing
effects. This a problem for low order schemes, which suffer from excessive
dissipation, but also for high order schemes, which rely on regularity
assumptions that may only hold piecewise on the exact solution.
Moreover, in the inviscid nonlinear case, even smooth initial boundary
data can develop discontinuities in finite time due to wave-breaking phenomena.

And finally, a convergent scheme can still produce unsatisfactory results
from a qualitative point of view, because it may not enforce exact
conservation at the discrete level, or it may produce spurious oscillations
around sharp features, or it may even converge to the wrong kind of weak
solution (violation of entropy conditions).

In the master’s thesis we’ve tackled these problems in 2D using
well-established but modern techniques that combine the finite volume
method (FVM) with Godunov’s method and weighted essentially
non-oscillatory (WENO) reconstructions using least-squares polynomial
interpolation, following the works of Hu and Shu and others.
We’ve successfully generalized type-I WENO schemes to polygonal
meshes and have shown their effectiveness on Voronoi tessellations.

This is in line with the modern trend towards more flexibility in the
discretization of problems’ domains. The main benefits are the
following: greater freedom in the meshing process, easier handling
of complex geometries, straightforward local refinement (no hanging
nodes) and natural treatment of hybrid grids and non-matching
interfaces without overlapping patches.