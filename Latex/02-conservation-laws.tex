\chapter{Systems of conservation laws}  \label{ch:conservation-laws}

%Compressible Euler equations are better understood in the more general framework
%of \emph{hyperbolic systems of conservation laws}. The same framework
%is also one of the most general ones to which the numerical methods
%of Chapters 3 and 4 can be applied without substantial changes,
%hence it deserves an adequate introduction. In the next chapter,
%the basic results of the theory of hyperbolic systems of conservation laws
%will be presented.

The theory of \emph{hyperbolic systems of conservation laws}
is the branch of mathematical analysis that investigates
partial differential equations of the form
\begin{equation} \label{eq:hscl-vectorial}
\partial_t \vec{u}(\vec{x},t) + \diver(\tns{F}(\vec{u}(\vec{x},t))) = 0,
\end{equation}
or generalizations thereof,
with $\vec{x} \in \Omega \subseteq \R^n$, $t \geq 0$ and
$\vec{u} \in \mathcal{U} \subseteq \R^m$.
$\tns{F}(\vec{u})$ is a (possibly) nonlinear function
known as \emph{flux}, which fully characterizes the equation and,
crucially, only depends on $\vec{u}$, and not any of its higher order
% derivatives $\nabla\vec{u}$, $\nabla^2\vec{u}$, etc.
derivatives: this means that equation
\eqref{eq:hscl-vectorial} is a first-order system.

The theory dates back to the works of Euler, who in the 1750s
first used the principle of conservation of mass and momentum
to describe the motion of an ideal fluid.
Since thermodynamics was still in its infancy at the time,
Euler and other scholars worked under the assumption of incompressible
flows; it was only later that the first system of conservation laws was
obtained, by working under the assumption of barotropic flow
(i.e.\ that pressure in a gas is a function of its density only).
By the mid-19th century, the latter assumption was generalized to
a principle of conservation of energy, and the full compressible Euler
equations of Chapter 1 were obtained.

It is fair to say that, throughout the 19th century, the field
of hyperbolic systems of conservation laws (which at the time was just
analytical gas dynamics) and the field of thermodynamics evolved together.
By the end of the century, the study of conservation laws
had already found numerous applications in the mathematical modelling
of combustion, detonation, and aerodynamics, which are still
relevant to this day.
In the 1940s, all the existing theory was summarized and consolidated
in a highly influential book by Courant and Friedrichs
\cite{courant-friedrichs}; after that, works by Hopf and Lax~\cite{hopf,lax}
in the 1950s layed the foundations for the present state of the art
in the theory of hyperbolic systems of conservation laws,
using modern analytical techniques.

In the following 70 years, groundbreaking results were obtained
by authors such as Glimm, Kruzhkov, Bressan, just to name a few.
Nevertheless, the theory of hyperbolic systems of conservation laws is still
very far from being complete, as many fundamental open problems remain
unsolved: among them, the problem of global existence of weak
solutions in the $m,n > 1$ case and the problem of uniqueness of weak solutions.
%using some sort of generalization of the second principle of thermodynamics.
In the words of Constantine Dafermos, whose excellent monograph \cite{dafermos}
was used as a reference to write this chapter,
\begin{quoting}
\textit{The Cauchy problem in the large [\dots]
%may be considered only in the context of weak solutions. This
is still terra incognita for systems of more
than one equation in several space dimensions, as the analysis is at
present facing seemingly insurmountable obstacles. It may turn out that
the Cauchy problem is not generally well-posed, either because of
catastrophic failure of uniqueness, or because distributional solutions
fail to exist. In the latter case one would have to resort to the class
of weaker, measure-valued solutions. It is even conceivable that
hyperbolic systems should be perceived as mere shadows, in the Platonic sense,
of diffusive systems with minute viscosity or dispersion.}
\end{quoting}
In this chapter, we provide a brief overview of the theory of hyperbolic
systems of conservation laws, with a focus on the qualitative properties
of the solutions, the local well-posedness theory in the smooth~$m,n > 1$ case,
and everything else we consider relevant as a background for the development
of effective numerical schemes.
%Other books that proved useful were the ones by \dots

\section{Definitions and technical assumptions}

First of all, we shall make it clear how the divergence operator
is used in equation \eqref{eq:hscl-vectorial}.
To do so, we switch to tensorial notation and make everything explicit:
\begin{equation} \label{eq:hscl-tensorial}
\partial_t \vec{u}^i(\vec{x},t) + \partial_j (\tns{F}^{ij}(\vec{u}(\vec{x},t))) = 0
\quad \text{for each $i = 1, \dots, m$.}
\end{equation}
As a remainder, $\vec{x} \in \Omega \subseteq \R^n$, $t \geq 0$ and
$\vec{u} \in \mathcal{U} \subseteq \R^m$.
Once again, we refer to Appendix A for a brief introduction
to tensors and their notation.
Each of the $m$ unknowns $\vec{u}^i(\vec{x},t)$ is known as
\emph{conserved variable}, for reasons which will be explained shortly;
the whole vector $\vec{u}(\vec{x},t)$ at each point in space and time must belong
to an open set of admissible values $\mathcal{U}$.
The flux function $\tns{F}(\vec{u})$, whose domain is $\mathcal{U}$,
is a smooth tensor field with values in $\R^m \otimes \R^n$,
and the divergence operator is applied with respect
to its $j$ index, which is the one associated to the number
of spatial dimensions $n$ (whereas $i$ is associated to the number of
scalar conservation laws $m$).

By the chain rule, equation \eqref{eq:hscl-tensorial} is equivalent
for regular solutions to the quasilinear system
\begin{equation} \label{eq:hscl-quasilinear-tensorial}
\partial_t \vec{u}^i(\vec{x},t)
+  \partial_k \tns{F}^{ij} (\vec{u}(\vec{x},t))
\, \partial_j \vec{u}^k (\vec{x},t) = 0
\quad \text{for each $i = 1, \dots, m$.}
\end{equation}
If, for any fixed $j$, we denote the $m \times m$ Jacobian matrix
$\partial_k \tns{F}^{ij}$ by $d\tns{F}^j$, then the quasilinear system
can be expressed in the more readable vectorial notation as
\begin{equation} \label{eq:hscl-quasilinear-vectorial}
\partial_t \vec{u}(\vec{x},t)
+ \sum_{j=1}^n d\tns{F}^j(\vec{u}(\vec{x},t))
\, \partial_j \vec{u}(\vec{x},t) = 0.
\end{equation}
New we are ready to introduce the fundamental, albeit technical,
assumption of \emph{hyperbolicity}:

\begin{defi}[Hyperbolic system]
The quasilinear system \eqref{eq:hscl-quasilinear-vectorial},
and by extension system \eqref{eq:hscl-vectorial} in divergence form,
are \emph{hyperbolic} if, for all versors $\vec{\nu}$ in $\R^n$ and all state
vectors $\vec{u}$ in $\mathcal{U}$, the matrix
\[
d\mat{F}(\vec{\nu},\vec{u}) \deq \sum_{j=1}^n \vec{\nu}^j d\mat{F}^j(\vec{u})
\]
known as \emph{Jacobian of the flux in the direction of $\vec{\nu}$},
is diagonalizable with real eigenvalues
\[
\lambda_1(\vec{\nu},\vec{u}) \leq \dots \leq \lambda_m (\vec{\nu},\vec{u}).
\]
If the eigenvalues are distinct for all values of $\vec{\nu}$ and $\vec{u}$,
an assumption known as \emph{strong hyperbolicity},
or if their multiplicities are at least constant, an assumption
known as \emph{constant hyperbolicity}, then it is possible to
prove the local existence of smooth right-eigenvector fields
\[
R_1(\vec{\nu},\vec{u}), \dots, R_m (\vec{\nu},\vec{u}).
\]
\end{defi}



\clearpage

% definition of hyperbolicity
% uniform qualitative properties (elliptic have instantaneous propagation etc)
% motivation using well-posedness of linear problem
% motivation using plane waves
% simmetrizable
% convex entropy
% chain of implications

\subsection*{The case of 2D compressible Euler equations}

In the case of the 2D compressible Euler equations, we require that
mass density $\rho(\vec{x},t)$ and total energy density $e(\vec{x},t)$
are strictly positive, hence
$\mathcal{U} = (0,+\infty) \times \R^2 \times (0,+\infty)$.
These assumptions on the conserved variables also imply that
other physical quantities like temperature and pressure are strictly
positive.
% euler equations in quasilinear form
% eigenstructure of Euler equations (page 104 Toro)
% entropy / entropy flux pair, convexity

\section{Qualitative properties of the solutions}
% conservation
% plane waves + information propagates at a finite speed (theorem 4.1.1)
% breakdown of classical solutions (burgers)

\section{Weak solutions}
% definition of weak solution
% rankine-hugoniot
% weak solutions are not unique
% different kinds of discontinuities; definition of shocks
% no proof of convergence for numerical schemes

\section{Boundary conditions}
% eh allora, praticamente











